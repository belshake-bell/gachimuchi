% ^^A -*- japanese-latex -*-
%
\ifx\epTeXinputencoding\undefined\else
  \epTeXinputencoding utf8 % ^^A added (2017-10-04)
\fi
%
% \iffalse
%
%   gachimuchi.dtx
%
%<*driver>
\documentclass[uplatex,a4paper,12Q,dvipdfmx]{jsarticle}
\usepackage{doc}
\usepackage{metalogo}
\providecommand*{\eTeX}{$\varepsilon$-\TeX}
\providecommand*{\pLaTeX}{p\kern-.05em\LaTeX}
\providecommand*{\pLaTeXe}{p\kern-.05em\LaTeXe}
\providecommand*{\upTeX}{u\pTeX}
\providecommand*{\XeTeX}{XeTeX}
\usepackage{ifptex}
\ifptex
  \xspcode"5C=1 %% \
  \xspcode"22=1 %% "
\fi
\addtolength{\textwidth}{-1.2in}
%% \addtolength{\evensidemargin}{1.5in}
\addtolength{\oddsidemargin}{1.5in}
%% \addtolength{\marginparwidth}{1.5in}
\setlength\marginparpush{0pt}
\usepackage{etoolbox}
\usepackage{listings}
\usepackage{color}
\usepackage{xspace}
\usepackage{ifpdf}
\usepackage{hyperref}
\ifptex
  \usepackage{pxjahyper}
\fi
\usepackage[T1]{fontenc}
\usepackage{lmodern}
\definecolor{myblue}{rgb}{0,0,0.75}
\definecolor{mygreen}{rgb}{0,0.45,0}
\hypersetup{colorlinks,
            hyperfootnotes=false,
            linkcolor=myblue,
            urlcolor=mygreen}
\CodelineNumbered
\EnableCrossrefs
% \CodelineIndex
\RecordChanges                  % Gather update information
\setcounter{StandardModuleDepth}{1}

\usepackage{gachimuchimacro}

\makeatletter

\newrobustcmd*{\file}[1]{\mbox{\ttfamily#1}}
\newrobustcmd*{\pkg}[1]{\mbox{\sffamily#1}}
\newcommand*{\Lcount}[1]{\textsl{\small#1}}
\newcommand*{\Lopt}[1]{\textsf{#1}}
\newcounter{@clineno}
\def\mlineplus#1{\setcounter{@clineno}{\arabic{CodelineNo}}%
   \addtocounter{@clineno}{#1}\arabic{@clineno}}
\def\cmd#1{\cs{\expandafter\cmd@to@cs\string#1}}
\def\cmd@to@cs#1#2{\char\number`#2\relax}
\DeclareRobustCommand{\cs}[1]{\texttt{\char`\\#1}}
\DeclareRobustCommand{\env}[1]{\texttt{#1}}
\providecommand{\marg}[1]{%
  {\ttfamily\char`\{}\meta{#1}{\ttfamily\char`\}}}
\providecommand{\oarg}[1]{%
  {\ttfamily[}\meta{#1}{\ttfamily]}}
\def\TODO{\par\textsf{To Do:\ }}%
\makeatother

\GetFileInfo{gachimuchimacro.dtx}
\begin{document}
  \DocInput{gachimuchimacro.dtx}
\end{document}
%</driver>
%
% \fi
% \changes{v1.0}{2019/10/29}
%   {make this .dtx file}
% \iffalse TeX commands \fi
% \DoNotIndex{\ ,\-,\/}
% \DoNotIndex{\above,\abovedisplayshortskip,\abovedisplayskip}
% \DoNotIndex{\abovewithdelims,\accent,\adjdemerits}
% \DoNotIndex{\advance,\afterassignment,\aftergroup}
% \DoNotIndex{\atop,\atopwithdelims}
% \DoNotIndex{\badness,\baselineskip,\batchmode}
% \DoNotIndex{\begingroup,\belowdisplayshortskip,\belowdisplayskip}
% \DoNotIndex{\binoppenalty,\botmark,\box}
% \DoNotIndex{\boxmaxdepth,\brokenpenalty}
% \DoNotIndex{\catcode,\char,\chardef}
% \DoNotIndex{\cleaders,\closein,\closeout}
% \DoNotIndex{\clubpenalty,\copy,\count}
% \DoNotIndex{\countdef,\cr,\crcr,\csname}
% \DoNotIndex{\day,\cr,\crcr,\csname}
% \DoNotIndex{\countdef,\deadcycles,\def,\defaulthyphenchar}
% \DoNotIndex{\defaultskewchar,\delcode,\delimiter,\delimiterfactor}
% \DoNotIndex{\delimitershortfall,\dimen,\dimendef,\discretionary}
% \DoNotIndex{\displayindent,\displaylimits,\displaystyle,\displaywidowpenalty}
% \DoNotIndex{\displaywidth,\divide,\doublehyphendemerits,\dp,\dump}
% \DoNotIndex{\edef,\else,\emergencystretch,\end}
% \DoNotIndex{\endcsname,\endgroup,\endinput,\endlinechar}
% \DoNotIndex{\eqno,\errhelp,\errmessage,\errorcontextlines}
% \DoNotIndex{\errorstopmode,\escapechar,\everycr,\everydisplay}
% \DoNotIndex{\everyhbox,\everyjob,\everymath,\everypar}
% \DoNotIndex{\everyvbox,\exhyphenpenalty,\expandafter}
% \DoNotIndex{\fam,\fi,\finalhyphendemerits,\firstmark}
% \DoNotIndex{\floatingpenalty,\font,\fontdimen,\fontname,\futurelet}
% \DoNotIndex{\gdef,\global,\globaldefs}
% \DoNotIndex{\halign,\hangafter,\hangindent,\hbadness}
% \DoNotIndex{\hbox,\hfil,\hfill,\hfilneg}
% \DoNotIndex{\hfuzz,\hoffset,\holdinginserts,\hrule}
% \DoNotIndex{\hsize,\hskip,\hss,\ht}
% \DoNotIndex{\hyphenation,\hyphenchar,\hyphenpenalty}
% \DoNotIndex{\if,\ifcase,\ifcat,\ifdim}
% \DoNotIndex{\ifeof,\iffalse,\ifhbox,\ifhmode}
% \DoNotIndex{\ifinner,\ifmmode,\ifnum,\ifodd}
% \DoNotIndex{\iftrue,\ifvbox,\ifvmode,\ifvoid}
% \DoNotIndex{\ifx,\ignorespaces,\immediate,\indent}
% \DoNotIndex{\input,\inputlineno,\insert,\insertpenalties,\interlinepenalty}
% \DoNotIndex{\jobname}
% \DoNotIndex{\kern}
% \DoNotIndex{\language,\lastbox,\lastkern,\lastpenalty}
% \DoNotIndex{\lastskip,\lccode,\leaders,\left}
% \DoNotIndex{\lefthyphenmin,\leftskip,\leqno,\let}
% \DoNotIndex{\limits,\linepenalty,\lineskip,\lineskiplimit}
% \DoNotIndex{\long,\looseness,\lower,\lowercase}
% \DoNotIndex{\mag,\mark,\mathaccent,\mathbin}
% \DoNotIndex{\mathchar,\mathchardef,\mathchoice,\mathclose}
% \DoNotIndex{\mathcode,\mathinner,\mathop,\mathopen}
% \DoNotIndex{\mathord,\mathpunct,\mathrel,\mathsurround}
% \DoNotIndex{\maxdeadcycles,\maxdepth,\meaning,\medmuskip}
% \DoNotIndex{\message,\mkern,\month,\moveleft}
% \DoNotIndex{\moveright,\mskip,\multiply,\muskip,\muskipdef}
% \DoNotIndex{\newlinechar,\noalign,\noboundary,\noexpand}
% \DoNotIndex{\noindent,\nolimits,\nonscript,\nonstopmode}
% \DoNotIndex{\nulldelimiterspace,\nullfont,\number}
% \DoNotIndex{\omit,\openin,\openout,\or}
% \DoNotIndex{\outer,\output,\outputpenalty,\over}
% \DoNotIndex{\overfullrule,\overline,\overwithdelims}
% \DoNotIndex{\pagedepth,\pagefilllstretch,\pagefillstretch,\pagefilstretch}
% \DoNotIndex{\pagegoal,\pageshrink,\pagestretch,\pagetotal}
% \DoNotIndex{\par,\parfillskip,\parindent,\parshape}
% \DoNotIndex{\parskip,\patterns,\pausing,\penalty}
% \DoNotIndex{\postdisplaypenalty,\predisplaypenalty,\predisplaysize}
% \DoNotIndex{\pretolerance,\prevdepth,\prevgraf}
% \DoNotIndex{\radical,\raise,\read,\relax}
% \DoNotIndex{\relpenalty,\right,\righthyphenmin,\rightskip,\romannumeral}
% \DoNotIndex{\scriptfont,\scriptscriptfont,\scriptscriptstyle,\scriptspace}
% \DoNotIndex{\scriptstyle,\scrollmode,\setbox,\setlanguage}
% \DoNotIndex{\sfcode,\shipout,\show,\showbox}
% \DoNotIndex{\showboxbreadth,\showboxdepth,\showlists,\showthe}
% \DoNotIndex{\skewchar,\skip,\skipdef,\spacefactor}
% \DoNotIndex{\spaceskip,\span,\special,\splitbotmark}
% \DoNotIndex{\splitfirstmark,\splitmaxdepth,\splittopskip,\string}
% \DoNotIndex{\tabskip,\textfont,\textstyle,\the}
% \DoNotIndex{\thickmuskip,\thinmuskip,\time,\toks}
% \DoNotIndex{\toksdef,\tolerance,\topmark,\topskip}
% \DoNotIndex{\tracingcommands,\tracinglostchars,\tracingmacros,\tracingonline}
% \DoNotIndex{\tracingoutput,\tracingpages,\tracingparagraphs,\tracingrestores}
% \DoNotIndex{\tracingstats}
% \DoNotIndex{\uccode,\uchyph,\underline,\unhbox}
% \DoNotIndex{\unhcopy,\unkern,\unpenalty,\unskip}
% \DoNotIndex{\unvbox,\unvcopy,\uppercase}
% \DoNotIndex{\vadjust,\valign,\vbadness,\vbox}
% \DoNotIndex{\vcenter,\vfil,\vfill,\vfilneg}
% \DoNotIndex{\vfuzz,\voffset,\vrule,\vsize}
% \DoNotIndex{\vskip,\vsplit,\vss,\vtop}
% \DoNotIndex{\wd,\widowpenalty,\write}
% \DoNotIndex{\xdef,\xleaders,\xspaceskip}
% \DoNotIndex{\year}

% \iffalse e-TeX commands \fi
% \DoNotIndex{\protected,\detokenize,\unexpanded}
% \DoNotIndex{\readline,\scantokens}
% \DoNotIndex{\eTeXrevision,\eTeXversion,\currentgrouplevel,\currentgrouptype}
% \DoNotIndex{\ifcsname,\ifdefined,\lastnodetype}
% \DoNotIndex{\marks}
% \DoNotIndex{\beginL,\beginR,\endL,\endR}
% \DoNotIndex{\TeXXeTstate,\predisplaydirection}
% \DoNotIndex{\interactionmode,\showgroups,\showtokens}
% \DoNotIndex{\tracingscantokens,\tracinggroups}
% \DoNotIndex{\everyeof,\middle,\unless}
%
% \def\Omaginai{おまじないとか前提スタイル・ファイルとか}
% \def\TemporalVar{局所変数定義}
% \def\Commandlabel{ラベル定義}
% \def\Gakufu{楽譜環境}
% \def\Carret{キャレット付き数字表記}
% \def\Accidental{臨時記号}
% \def\Subscription{添え字}
% \def\UsefulMacros{便利なマクロ}
% \def\ChordSymbols{コードシンボル}
% \def\GeidaiSymbols{芸大和声の表記}
% \def\ZCHAR{楽譜上の文字}
% \def\ZCHORD{和声記号配置}
% \def\FigureBass{数字付き低音}
% \def\Tenni{転位記号}
% \def\Harmonie{ドイツ式和声記号}
% \def\ChordProg{Chord Progressionの表記}
% \def\MusicSymbols{譜面上の記号}
% \def\Resets{初期設定}
%
% \title{gachimuchi}
% \author{bell}
% \date{\filedate}
% \maketitle
% \tableofcontents
%
% \MakeShortVerb{\|}
%
% \section{はじめに}
% \section{概要}
% \subsection{\Omaginai}
% \subsection{\TemporalVar}
% \subsection{\Commandlabel}
% \subsection{\Gakufu}
% \subsection{\Carret}
% \subsection{\Accidental}
% \subsection{\Subscription}
% \subsection{\UsefulMacros}
% \subsection{\ChordSymbols}
% \subsection{\GeidaiSymbols}
% \subsection{\ZCHAR}
% \subsection{\ZCHORD}
% \subsection{\FigureBass}
% \subsection{\Tenni}
% \subsection{\Harmonie}
% \subsection{\ChordProg}
% \subsection{\MusicSymbols}
% \subsection{\Resets}
%
%
% \StopEventually{}
% \clearpage
% \section{実装}
% \changes{v1.1}{2020/01/23}
%    {依存部分を減らし,また条件分けすることで\pLaTeX を
%     前提としなくした(不完全).}
%
%    \begin{macrocode}
%<*macro>
%%=============================================================================%
%%                                                                             %
%%   GGGGG          A          CCCCC    H       H    IIIII                     %
%%  G     G        A A        C     C   H       H      I                       %
%% G       G      A   A      C       C  H       H      I                       %
%% G             A     A     C          H       H      I                       %
%% G            A       A    C          HHHHHHHHH      I     --------          %
%% G     GGGG   AAAAAAAAA    C          H       H      I                       %
%% G       G    A       A    C       C  H       H      I                       %
%%  G     GG    A       A     C     C   H       H      I                       %
%%   GGGGG G    A       A      CCCCC    H       H    IIIII                     %
%%                                                                             %
%%                M         M   UUU     UUU    CCCCC    H       H    IIIII     %
%%                MM       MM    U       U    C     C   H       H      I       %
%%                M M     M M    U       U   C       C  H       H      I       %
%%                M M     M M    U       U   C          H       H      I       %
%%                M  M   M  M    U       U   C          HHHHHHHHH      I       %
%%                M  M   M  M    U       U   C          H       H      I       %
%%                M   M M   M    U       U   C       C  H       H      I       %
%%                M   M M   M     U     U     C     C   H       H      I       %
%%                M    M    M      UUUUU       CCCCC    H       H    IIIII     %
%%                                                                             %
%%=============================================================================%
%\NeedsTeXFormat{pLaTeX2e}% -bell 2020-01-23
\NeedsTeXFormat{LaTeX2e}% +bell 2020-01-23
\ProvidesPackage{gachimuchimacro}
  [2020/07/11 Gachimuchi Koza Macro set ver. 1.2]
% [2020/04/20 Gachimuchi Koza Macro set]
%    \end{macrocode}
% \subsection{\Omaginai}
% いくつかのスタイル・ファイルを前提として読み込んでいます.
% \begin{itemize}
% \item \file{amssymb}:
%   \txSquare のため.
% \item \file{noconflict}:
%   競合するコマンドを退避するため.
% \item \file{multirow}:
%   多分使ってないので削除できそう.
% \item \file{etoolbox}:
%   確かコマンドを使っていると思うので.
% \item \file{graphicx}:
%   文字の拡大・縮小のため.
% \item \file{bxghost}:
%   ゴースト文字のため.
% \item \file{tikz}:
%   様々な描画のため.
% \item \file{hyperref}:
%   \cmd{\texorpdfstring}のため.
%   これさえどうにかなればなぁ.
% \item \file{ifptex}:
%   \pTeX 系列を判定するため.
% \item \file{pxjahyper}(\pTeX 系列のときのみ):
%   \file{hyperref}のパッチのため.
% \item \file{everyhook}:
%   Hookを設定するため.
% \item \file{xkeyval}:
%   Key-val方式でのオプションを与えるコマンドのため.
% \item \file{musixtex}:
%   楽譜のため.
% \item \file{musixdbr},\ \file{musixper},\ \file{musixlmd}
% \end{itemize}
% また,\file{biblatex}と\file{musixtex}は\cmd{\addspace}命令で,
% \changes{v1.1}{2020/01/23}
%    {\file{luatexja}と\file{musixtex}の競合の解決}
% \file{luatexja}と\file{musixtex}は\cmd{\zh}命令および\cmd{\zw}命令で
% 競合を起こすので,
% \file{musixtex}の命令を\env{Music}環境内でのみ戻すことで対処しています.
%
%    \begin{macrocode}
%%==============================================================================
%% 依存パッケージの読み込み
%%==============================================================================
\RequirePackage{amssymb}
\RequirePackage{noconflict}
\RequirePackage{multirow}
\RequirePackage{etoolbox}
\RequirePackageWithOptions{graphicx}
% \RequirePackage{pxghost}% -bell 2020-01-23
\RequirePackage{bxghost}% +bell 2020-01-23
\RequirePackage{tikz}
\RequirePackageWithOptions{hyperref}
\RequirePackage{ifptex}
\ifptex
  \RequirePackageWithOptions{pxjahyper}
\fi
\RequirePackage{everyhook}
\RequirePackage{xkeyval}

\@ifpackageloaded
  {biblatex}%
  {\message{Loaded biblatex^^J}\save{biblatex}{addspace}}%
  {}

\ifdefined\zw
  \message{defined \noexpand\zw^^J}%
  \let\bell@zw\zw \let\bell@zh\zh
  \save{bell@otherpkg}{zw,zh}%
\fi% +bell 2020-01-23

\ifptex
% bellmacros.styと同じ定義
  \def\bell@zw{zw}
  \def\bell@zh{zh}
\fi

\RequirePackage{musixtex}
\input{musixdbr}
\input{musixper}
\input{musixlmd}
\save{musixtex}{addspace}
\save{musixtex}{zh}% +bell 2020-01-23
\save{musixtex}{zw}% +bell 2020-01-23

\@ifpackageloaded
  {biblatex}%
  {\restore{biblatex}}%
  {}

\ifdefined\bell@otherpkg@zw
  \message{redefined \noexpand\zw^^J}%
  \restore{bell@otherpkg}
\fi% +bell 2020-01-23

%    \end{macrocode}
% 
% \subsection{\TemporalVar}
%    \begin{macrocode}
%%==============================================================================
%% 局所変数定義
%%==============================================================================
%    \end{macrocode}
% \begin{macro}{\gcmc@test@temptoka}
% テンポラリなトークンレジスタを確保しています.
% 
%    \begin{macrocode}
\newtoks\gcmc@test@temptoka
\def\thr@@qu@rt{.75}
%    \end{macrocode}
% \end{macro}
% \subsection{\Commandlabel}
%    \begin{macrocode}
%%==============================================================================
%% ラベル定義
%%==============================================================================
%    \end{macrocode}
% \begin{macro}{\piecelabel}
% \changes{v1.1}{2020/01/21}
%    {\cmd{\Piecelabel}から\cmd{\piecelabel}へ}
% 楽譜環境用のラベルの命令を定義しています.
%    \begin{macrocode}
\newcommand{\piecelabel}[1]{\label{piece:#1}}
%    \end{macrocode}
% \end{macro}
% \begin{macro}{\pieceref}
% \changes{v1.1}{2020/01/21}
%    {\cmd{\Pieceref}から\cmd{\pieceref}へ}
% 楽譜環境用の参照の命令を定義しています.
% 
%    \begin{macrocode}
\newcommand{\pieceref}[1]{譜例\ref{piece:#1}}
%    \end{macrocode}
% \end{macro}
% \begin{macro}{\c@piececnt}
% \changes{v1.1}{2020/01/21}
%    {\cmd{\c@pieceNum}から\cmd{\c@piececnt}へ}
%    \begin{macrocode}
\newcounter{piececnt}
%    \end{macrocode}
% \end{macro}
% \begin{macro}{\thepiececnt}
% \changes{v1.1}{2020/01/21}
%    {\cmd{\thepieceNum}から\cmd{\thepiececnt}へ}
% 同時に使用されるであろうカウンタを設定します.
%    \begin{macrocode}
\renewcommand{\thepiececnt}{\arabic{piececnt}}
%    \end{macrocode}
% \end{macro}
% \begin{macro}{\subpiecelabel}
% \changes{v1.2}{2020/07/11}
%    {楽譜番号の小区分のために\cmd{\subpiecelabel}を実装.}
% 小区分した楽譜環境用のラベルの命令を定義しています.
%    \begin{macrocode}
\newcommand{\subpiecelabel}[1]{\label{piece:#1}}
%    \end{macrocode}
% \end{macro}
% \begin{macro}{\subpieceref}
% \changes{v1.2}{2020/07/11}
%    {楽譜番号の小区分のために\cmd{\subpieceref}を実装.}
% 小区分した楽譜環境用の参照の命令を定義しています.
% 
%    \begin{macrocode}
\newcommand{\subpieceref}[1]{譜例\ref{piece:#1}}
%    \end{macrocode}
% \end{macro}
% \begin{macro}{\c@subpiececnt}
% \changes{v1.2}{2020/07/11}
%    {楽譜番号の小区分のために\cmd{\c@subpiececnt}を実装.}
% 小区分した楽譜環境用のカウンタを設定します.
%    \begin{macrocode}
\newcounter{subpiececnt}
%    \end{macrocode}
% \end{macro}
% \begin{macro}{\thesubpiececnt}
% \changes{v1.2}{2020/07/11}
%    {楽譜番号の小区分のために\cmd{\thesubpiececnt}を実装.}
% 小区分した楽譜環境用のカウンタを出力するコマンドを定義します.
%    \begin{macrocode}
\renewcommand{\thesubpiececnt}{\thepiececnt(\alph{subpiececnt})}
%    \end{macrocode}
% \end{macro}
% \subsection{\Gakufu}
% \file{musixtex}に依存しています.
% 
%    \begin{macrocode}
%%==============================================================================
%% 楽譜番号
%%==============================================================================
%    \end{macrocode}
% \begin{macro}{\piece@num}
% \cmd{\c@piececnt}カウンタを進め,引数を印刷してカウンタを表示します.
%    \begin{macrocode}
\newcommand{\piece@num}[1]{%
  \refstepcounter{piececnt}#1\thepiececnt
  \xdef\bell@refcnt{\@currentlabel}%
}
%    \end{macrocode}
% \end{macro}
% \begin{macro}{\ifprintpiecenumber}
% \changes{v1.2}{2020/07/11}
%    {\cmd{\bell@piecenumbers},\cmd{\nopiecenumbers},\cmd{piecenumbers}を
%     \textbf{削除}.
%     \cmd{\ifprintpiecenumber}を代替として定義.}
%    \begin{macrocode}
\newif\ifprintpiecenumber

%    \end{macrocode}
% \end{macro}
%    \begin{macrocode}
%%==============================================================================
%% 楽譜環境(musixtex依存)
%%==============================================================================
%    \end{macrocode}
% \begin{macro}{\bell@resolve@musixtex@conflicts}
% \changes{v1.1}{2020/01/23}
%    {\file{luatexja}と\file{musixtex}の競合の解決}
% \cmd{\addspace}命令,\cmd{\zh}命令および\cmd{\zw}命令を
% \file{musixtex}のものに戻します.
%    \begin{macrocode}
\def\bell@resolve@musixtex@conflicts{% +bell 2020-01-23
  \let\addspace\musixtexaddspace% <-biblatexとのマクロ競合回避のため
  \let\zh\musixtexzh \let\zw\musixtexzw% +bell 2020-01-23
}
%    \end{macrocode}
% \end{macro}
% \begin{macro}{\bell@envMusic@default}
% 楽譜環境の先頭で初期化するための諸々を入れてます.
% \changes{v1.1}{2020/01/23}
%    {\file{luatexja}と\file{musixtex}の競合の解決.}
% \changes{v1.1}{2020/01/23}
%    {\file{bxjaprnind}対策にhookを設定.}
% \changes{v1.2}{2020/07/11}
%    {\cmd{\bell@piecenumbers},\cmd{\nopiecenumbers},\cmd{piecenumbers}を
%     \textbf{削除}.
%     \cmd{\ifprintpiecenumber}を代替として使用する.}
% またここで\cmd{\addspace}命令,\cmd{\zh}命令および\cmd{\zw}命令を
% \file{musixtex}のものに戻しています.
%    \begin{macrocode}
\def\bell@envMusic@default#1{%
  \parindent\z@
  \y@\linewidth \advance\y@-#1 \y@\h@lf\y@
  \leftmargin\y@ \rightmargin\y@
  \advance\linewidth -\rightmargin
  \advance\linewidth -\leftmargin
  \advance\@totalleftmargin \leftmargin
  \printpiecenumbertrue
  \nobarnumbers
  \SavePostHook{par}{\bell@reserved@MusicPostHook}% +bell 2020-01-29
  \ClearPostHook{par}% +bell 2020-01-29
  \PushPostHook{par}{\parshape\@ne \@totalleftmargin\linewidth \ignorespaces}%
  \def\comp@internote{%
    \comput@specifics
    \n@loop
    \internote\csname staffspacing\romannumeral\noinstrum@nt\endcsname\Internote
    \repeat
    %\line@width\hsize% <-これがアレっぽそう
    \setlength{\line@width}{\linewidth}% <-変えてみた
  }% \let\addspace\musixtexaddspace% -bell 2020-01-23
  \bell@resolve@musixtex@conflicts% +bell 2020-01-23
}
\def\bell@envMusic@restore{% +bell 2020-01-29
% \RestorePostHook{par}{\bell@reserved@MusicPostHook}%
}
%    \end{macrocode}
% \end{macro}
% \begin{environment}{Music}
% 楽譜環境です.
% 上下に余白を設けています.
% 先に定義した\cmd{\bell@envMusic@default}を呼び出しています.
% 引数は楽譜の横幅で,規定値は|.85\linewidth|としています.
% \env{music}環境を内部的に呼び出してます.
%    \begin{macrocode}
\newenvironment{Music}[1][.85\linewidth]{%
  \par\penalty\@M\vspace{0.5\Cvs}%
  \bell@envMusic@default{#1}%
  \begin{music}}
 {\end{music}%
  \bell@envMusic@restore% +bell 2020-01-29
  \vspace{\h@lf\Cvs}\par}
%    \end{macrocode}
% \end{environment}
% \paragraph{floatMusic}\oarg{key-val opt.}\par
% floatな楽譜環境を定義します.
%
% まずオプション引数をKey-val式で与えられるようにします.
% \begin{description}
% \item[\meta{pos}] floatの位置を指定します. 規定は\Lopt{ht}.
% \item[\meta{width}] 楽譜の横幅を指定します. 規定は|.85\linewidth|.
% \end{description}
%    \begin{macrocode}

\define@cmdkey[bell]{floatMusic}{pos}[ht]{}
\define@cmdkey[bell]{floatMusic}{width}[.85\linewidth]{}
\presetkeys[bell]{floatMusic}{pos,width}{}

%    \end{macrocode}
% \begin{macro}{\c@scorefigure}
% float楽譜環境のためにカウンタを定義します.
%    \begin{macrocode}
\ifx\c@chapter\undefined
  \newcounter{scorefigure}
\else
  \newcounter{scorefigure}[chapter]
\fi
%    \end{macrocode}
% \end{macro}
% \begin{macro}{\thescorefigure}
%    \begin{macrocode}
\renewcommand{\thescorefigure}
     {\ifnum \c@chapter>\z@ \thechapter.\fi \@arabic\c@scorefigure}
%    \end{macrocode}
% \end{macro}
% \begin{macro}{\fps@scorefigure}
%    \begin{macrocode}
\def\fps@scorefigure{tbp}
%    \end{macrocode}
% \end{macro}
% \begin{macro}{\ftype@scorefigure}
%    \begin{macrocode}
\def\ftype@scorefigure{1}
%    \end{macrocode}
% \end{macro}
% \begin{macro}{\ext@scorefigure}
%    \begin{macrocode}
\def\ext@scorefigure{lop}
%    \end{macrocode}
% \end{macro}
% \begin{macro}{\fnum@scorefigure}
%    \begin{macrocode}
\def\fnum@scorefigure{\scorefigure\nobreak\thefigure}
%    \end{macrocode}
% \end{macro}
% \begin{macro}{\bell@scorefig}
%    \begin{macrocode}
\def\bell@scorefig{\@float{scorefigure}}
%    \end{macrocode}
% \end{macro}
% \begin{environment}{floatMusic}\oarg{key-val opt.}\par
%    \begin{macrocode}
\newenvironment{floatMusic}[1][]{%
  \setkeys[bell]{floatMusic}{#1}%
  \expandafter\bell@scorefig\expandafter[\cmdbell@floatMusic@pos]%
  \bell@envMusic@default{\cmdbell@floatMusic@width}%
  \begin{music}%
}{%
  \end{music}%
  \bell@envMusic@restore% +bell 2020-01-29
  \end@float}

%    \end{macrocode}
% \end{environment}
% \begin{macro}{\piecetitle}\oarg{composer}\marg{piece title}\par
% 楽譜のタイトルを表示します.
%    \begin{macrocode}
\newcommand*{\piecetitle}[2][\empty]{%
%  \begin{center}#2\end{center}\vspace{-1ex}%
  \par\begingroup\centering #2\par\endgroup\vspace{-1ex}%
  \def\reserved@a{#1}%
  \ifx\empty\reserved@a\else
    \hfill\begingroup #1\par\endgroup
  \fi\vspace{1ex}%
}

%    \end{macrocode}
% \end{macro}
% \begin{macro}{\Piecetitle}\oarg{composer}\marg{piece title}\par
% タイトルを\cmd{\huge}\cmd{\sffamily}で表示する\cmd{\piecetitle}です.
% もしかしたら\cmd{\piecetitle}の|*|変種でもいいかもしれません.
%    \begin{macrocode}
\newcommand*{\Piecetitle}[2][\empty]{\piecetitle[#1]{\huge\sffamily #2}}%
%    \end{macrocode}
% \end{macro}
% \begin{macro}{\piececaption}
% \begin{macro}{\@piececaption}
% \cmd{\piececaption}\oarg{not used}\marg{caption}の形で
% 楽譜環境にキャプションをつけます.
% オプション引数は使われている形跡がありません……どうしようとしたのでしょう?
%    \begin{macrocode}
\newcommand*{\piececaption}{\@dblarg{\@piececaption}}
\def\@piececaption[#1]#2{%
  \addcontentsline{lop}{piece}%
    {\protect\numberline{\thepiececnt}{\ignorespaces #2}}%
  \par\nopagebreak\begingroup
  \centering 譜例 \thepiececnt: #2\par
  \endgroup\vspace{-1ex}%
}

%    \end{macrocode}
% \end{macro}
% \end{macro}
% \begin{macro}{\notesintext}\marg{music expr.}\par
% インラインで音符などの構成要素のみを表示するマクロを定義します.
%    \begin{macrocode}
\newcommand{\notesintext}[1]{%
  \bgroup\let\extractline\relax%
  \bell@resolve@musixtex@conflicts
  \setlines{1}{0}\smallmusicsize \nobarnumbers \nostartrule%
  \staffbotmarg0pt \setclefsymbol1\empty \global\clef@skip0pt \relax%
  \startextract\addspace{-\afterruleskip}#1\zendextract%
  \egroup
}%

%    \end{macrocode}
% \end{macro}
% \begin{environment}{staffintext}
% インラインで楽譜を表示する環境を定義します.
%    \begin{macrocode}
\newenvironment{staffintext}{%
  \bgroup\let\extractline\relax%
  \bell@resolve@musixtex@conflicts
  \smallmusicsize \nobarnumbers \nostartrule%
  \staffbotmarg0pt \relax%
  \startextract\addspace{-\afterruleskip}%
}{%
  \zendextract\egroup%
}

%    \end{macrocode}
% \end{environment}
% \begin{macro}{\writebarno}
% \cmd{\writebarno}を再定義して,描画のフォーマットを指定しています.
%    \begin{macrocode}
\def\writebarno{\textbf{\small\the\barno\barnoadd}}

%    \end{macrocode}
% \end{macro}
% \begin{macro}{\listofpiece}
%    \begin{macrocode}
\newcommand{\listofpiece}{%
  \if@twocolumn\@restonecoltrue\onecolumn
  \else\@restonecolfalse\fi
  \chapter*{\listpiecename}%
  \@mkboth{\listpiecename}{}%
  \@starttoc{lop}%
  \if@restonecol\twocolumn\fi
}
%    \end{macrocode}
% \end{macro}
% \begin{macro}{\listpiecename}
%    \begin{macrocode}
\def\listpiecename{譜例}
%    \end{macrocode}
% \end{macro}
% \begin{macro}{\l@piece}
%    \begin{macrocode}
\newcommand*{\l@piece}{\@dottedtocline{1}{1zw}{3.683zw}}
%    \end{macrocode}
% \end{macro}
% \subsection{\Carret}
%    \begin{macrocode}
%%==============================================================================
%% キャレット付き数字表記
%%==============================================================================
%    \end{macrocode}
% \begin{macro}{\CrtNum}\marg{number}\par
% キャレットをつけた数字を出力します.
% アメリカでは単音の音高を度数で表す際に用いるようです.
%    \begin{macrocode}
\DeclareRobustCommand{\CrtNum}[1]{\^{#1}}

%    \end{macrocode}
% \end{macro}
% \subsection{\Accidental}
%    \begin{macrocode}
%%==============================================================================
%% 臨時記号(musixtex依存)
%%==============================================================================
%    \end{macrocode}
% \begin{macro}{\qu@rtsmallvalue}
% \begin{macro}{\tw@smallvalue}
% \begin{macro}{\thr@@smallvalue}
% \begin{macro}{\h@lfsmallvalue}
% \begin{macro}{\@neh@lfsmallvalue}
% \begin{macro}{\tw@h@lfsmallvalue}
% \begin{macro}{\thr@@h@lfsmallvalue}
% 数字定数マクロを定義します.
%    \begin{macrocode}
\newcommand{\qu@rtsmallvalue}{.2}
\newcommand{\tw@smallvalue}{1.6}
\newcommand{\thr@@smallvalue}{2.4}
\newcommand{\h@lfsmallvalue}{.4}
\newcommand{\@neh@lfsmallvalue}{1.2}
\newcommand{\tw@h@lfsmallvalue}{2.0}
\newcommand{\thr@@h@lfsmallvalue}{2.8}

%    \end{macrocode}
% \end{macro}
% \end{macro}
% \end{macro}
% \end{macro}
% \end{macro}
% \end{macro}
% \end{macro}
% \begin{macro}{\accidentaloffs@tunit}
%    \begin{macrocode}
\newdimen\accidentaloffs@tunit
%    \end{macrocode}
% \end{macro}
% \begin{macro}{\calc@accidentaloffs@tunit}
% \cmd{\accidentaloffs@tunit}に臨時記号のオフセット単位長を代入します.
%    \begin{macrocode}
\newcommand{\calc@accidentaloffs@tunit}{%
  \accidentaloffs@tunit \qu@rtsmallvalue \dimexpr \qn@width \relax
}

%    \end{macrocode}
% \end{macro}
% \begin{macro}{\Flat}
% \begin{macro}{\dFlat}
% \begin{macro}{\Sharp}
% \begin{macro}{\dSharp}
% \begin{macro}{\Natural}
%    \begin{macrocode}
\newcommand{\Flat}  {\nobreak\supsc{\txFlat}\nobreak}
\newcommand{\dFlat} {\nobreak\supsc{\txdFlat}\nobreak}
\newcommand{\Sharp} {\nobreak\supsc{\txSharp}\nobreak}
\newcommand{\dSharp}{\nobreak\supsc{\txdSharp}\nobreak}
\newcommand{\Natural}{\nobreak\supsc{\txNatural}\nobreak}
%    \end{macrocode}
% \end{macro}
% \end{macro}
% \end{macro}
% \end{macro}
% \end{macro}
% \begin{macro}{\txFlat}
% \begin{macro}{\txdFlat}
% \begin{macro}{\txSharp}
% \begin{macro}{\txdSharp}
% \begin{macro}{\txNatural}
%    \begin{macrocode}
\newcommand{\txFlat}{\ensuremath{\flat}}
\newcommand{\txSharp}{\ensuremath{\sharp}}
%\newcommand{\txdFlat}{\txFlat\hspace{-.1zw}\txFlat}%-bell 2020-01-23
\newcommand{\txdFlat}{\txFlat\hspace{-.1em}\txFlat}%+bell 2020-01-23
\newcommand{\txdSharp}{%
  %\settoheight{\@tempdima}{\musixchar53}%
  \settodepth{\@tempdimb}{\musixchar53}%
  %\advance\@tempdima\@tempdimb
  \eghostguarded{\raisebox{\@tempdimb}{\musixchar53}}%
}
\newcommand{\txNatural}{\ensuremath{\natural}}
%    \end{macrocode}
% \end{macro}
% \end{macro}
% \end{macro}
% \end{macro}
% \end{macro}
% \begin{macro}{\txCirc}
% \begin{macro}{\txTri}
% \begin{macro}{\txSqr}
%    \begin{macrocode}
\newcommand{\txCirc}{\ensuremath{\bigcirc}}
\newcommand{\txTri}{\ensuremath{\triangle}}
\newcommand{\txSqr}{\ensuremath{\square}}

%    \end{macrocode}
% \end{macro}
% \end{macro}
% \end{macro}
% \begin{macro}{\c@bell@shiftAccidentals}
% \begin{macro}{\iftok@remain}
%    \begin{macrocode}
\newcounter{bell@shiftAccidentals}
\newif\iftok@remain
\tok@remainfalse
%    \end{macrocode}
% \end{macro}
% \end{macro}
% \begin{macro}{\bell@getshift}\marg{offset chars}|\relax|\par
% 臨時記号のオフセットを指定して\cmd{\c@bell@shiftAccidentals}カウンタに代入します.
% \meta{offset chars}は\texttt{l, r, h, q}からなる文字列であり,
%    \begin{macrocode}
\def\bell@getshift#1\relax{%
  \setcounter{bell@shiftAccidentals}{0}%
  \@tempcnta0\relax%
  \def\@rgs{#1}%
  \loop
    \edef\t@ruc{\expandafter\f@tok \@rgs\relax\af@tok}%
    \edef\s@uite{\expandafter\s@tok \@rgs\empty\af@tok}%
    \ifcat a\t@ruc
      \n@i\expandafter`\t@ruc\relax
      \ifnum\n@i=`l\relax% left
        \ifnum\@tempcnta=0\relax
          \addtocounter{bell@shiftAccidentals}{4}%
        \else
          \addtocounter{bell@shiftAccidentals}{\@tempcnta}%
        \fi
        \@tempcnta0\relax%
      \fi
      \ifnum\n@i=`r\relax% right
        \ifnum\@tempcnta=0\relax
          \addtocounter{bell@shiftAccidentals}{-4}%
        \else
          \addtocounter{bell@shiftAccidentals}{-\@tempcnta}%
        \fi
        \@tempcnta0\relax%
      \fi
      \ifnum\n@i=`h\relax% half
        \advance\@tempcnta2\relax%
      \fi
      \ifnum\n@i=`q\relax% quart
        \advance\@tempcnta1\relax%
      \fi
    \fi
    \ifx\s@uite\empty \tok@remainfalse
    \else \tok@remaintrue
    \fi
    \iftok@remain \let\@rgs\s@uite%
  \repeat%
}

%    \end{macrocode}
% \end{macro}
% \begin{macro}{\na}
%    \begin{macrocode}
\let\orig@na\na
\renewcommand{\na}[2][]{%
  \bell@getshift#1\relax%
  \bgroup
  \edef\s@o@{\value{bell@shiftAccidentals}\accidentaloffs@tunit}%
  \offs@t{\orig@na{#2}}%
  \egroup
}
%    \end{macrocode}
% \end{macro}
% \begin{macro}{\fl}
%    \begin{macrocode}
\let\orig@fl\fl
\renewcommand{\fl}[2][]{%
  \bell@getshift#1\relax%
  \bgroup
  \edef\s@o@{\value{bell@shiftAccidentals}\accidentaloffs@tunit}%
  \offs@t{\orig@fl{#2}}%
  \egroup
}
%    \end{macrocode}
% \end{macro}
% \begin{macro}{\sh}
%    \begin{macrocode}
\let\orig@sh\sh
\renewcommand{\sh}[2][]{%
  \bell@getshift#1\relax%
  \bgroup
  \edef\s@o@{\value{bell@shiftAccidentals}\accidentaloffs@tunit}%
  \offs@t{\orig@sh{#2}}%
  \egroup
}
%    \end{macrocode}
% \end{macro}
% \begin{macro}{\dfl}
%    \begin{macrocode}
\let\orig@dfl\dfl
\renewcommand{\dfl}[2][]{%
  \bell@getshift#1\relax%
  \bgroup
  \edef\s@o@{\value{bell@shiftAccidentals}\accidentaloffs@tunit}%
  \offs@t{\orig@dfl{#2}}%
  \egroup
}
%    \end{macrocode}
% \end{macro}
% \begin{macro}{\dsh}
%    \begin{macrocode}
\let\orig@dsh\dsh
\renewcommand{\dsh}[2][]{%
  \bell@getshift#1\relax%
  \bgroup
  \edef\s@o@{\value{bell@shiftAccidentals}\accidentaloffs@tunit}%
  \offs@t{\orig@dsh{#2}}%
  \egroup
}

%    \end{macrocode}
% \end{macro}
% ここからobsoluteなコマンドです.
%    \begin{macrocode}
\newcommand{\hna}[1]{\loffset{\h@lfsmallvalue}{\na{#1}}}
\newcommand{\hfl}[1]{\loffset{\h@lfsmallvalue}{\fl{#1}}}
\newcommand{\hsh}[1]{\loffset{\h@lfsmallvalue}{\sh{#1}}}
\newcommand{\hdfl}[1]{\loffset{\h@lfsmallvalue}{\dfl{#1}}}
\newcommand{\hdsh}[1]{\loffset{\h@lfsmallvalue}{\dsh{#1}}}

\newcommand{\lhna}[1]{\loffset{\@neh@lfsmallvalue}{\na{#1}}}
\newcommand{\lhfl}[1]{\loffset{\@neh@lfsmallvalue}{\fl{#1}}}
\newcommand{\lhsh}[1]{\loffset{\@neh@lfsmallvalue}{\sh{#1}}}
\newcommand{\lhdfl}[1]{\loffset{\@neh@lfsmallvalue}{\dfl{#1}}}
\newcommand{\lhdsh}[1]{\loffset{\@neh@lfsmallvalue}{\dsh{#1}}}

\newcommand{\llna}[1]{\loffset{\tw@smallvalue}{\na{#1}}}
\newcommand{\llfl}[1]{\loffset{\tw@smallvalue}{\fl{#1}}}
\newcommand{\llsh}[1]{\loffset{\tw@smallvalue}{\sh{#1}}}
\newcommand{\lldfl}[1]{\loffset{\tw@smallvalue}{\dfl{#1}}}
\newcommand{\lldsh}[1]{\loffset{\tw@smallvalue}{\dsh{#1}}}

\newcommand{\llhna}[1]{\loffset{\tw@h@lfsmallvalue}{\na{#1}}}
\newcommand{\llhfl}[1]{\loffset{\tw@h@lfsmallvalue}{\fl{#1}}}
\newcommand{\llhsh}[1]{\loffset{\tw@h@lfsmallvalue}{\sh{#1}}}
\newcommand{\llhdfl}[1]{\loffset{\tw@h@lfsmallvalue}{\dfl{#1}}}
\newcommand{\llhdsh}[1]{\loffset{\tw@h@lfsmallvalue}{\dsh{#1}}}

\newcommand{\lllna}[1]{\loffset{\thr@@smallvalue}{\na{#1}}}
\newcommand{\lllfl}[1]{\loffset{\thr@@smallvalue}{\fl{#1}}}
\newcommand{\lllsh}[1]{\loffset{\thr@@smallvalue}{\sh{#1}}}
\newcommand{\llldfl}[1]{\loffset{\thr@@smallvalue}{\dfl{#1}}}
\newcommand{\llldsh}[1]{\loffset{\thr@@smallvalue}{\dsh{#1}}}

%    \end{macrocode}
% ここまでobsoluteなコマンドです.
%    \begin{macrocode}
\newcommand{\parna}[1]{\loffset{1.5}{\lpar{#1}\rpar{#1}}\loffset{.4}{\na{#1}}}
\newcommand{\parfl}[1]{\loffset{1.5}{\lpar{#1}\rpar{#1}}\loffset{.4}{\fl{#1}}}
\newcommand{\parsh}[1]{\loffset{1.5}{\lpar{#1}\rpar{#1}}\loffset{.4}{\sh{#1}}}
\newcommand{\pardfl}[1]{\loffset{2.0}{\lpar{#1}\rpar{#1}}\loffset{.4}{\dfl{#1}}}
\newcommand{\pardsh}[1]{\loffset{1.5}{\lpar{#1}\rpar{#1}}\loffset{.4}{\dsh{#1}}}

%    \end{macrocode}
%
%    \begin{macrocode}
\newcommand{\aFlat}{\nobreak\Flat}
\newcommand{\adFlat}{\nobreak\dFlat}
\newcommand{\aSharp}{\nobreak\Sharp}
\newcommand{\adSharp}{\nobreak\dSharp}
\newcommand{\aNatural}{\nobreak\Natural}
\newcommand{\bFlat}{\Flat\nobreak}
\newcommand{\bdFlat}{\dFlat\nobreak}
\newcommand{\bSharp}{\Sharp\nobreak}
\newcommand{\bdSharp}{\dSharp\nobreak}
\newcommand{\bNatural}{\Natural\nobreak}
\newcommand{\atxFlat}{\nobreak\txFlat}
\newcommand{\atxdFlat}{\nobreak\txdFlat}
\newcommand{\atxSharp}{\nobreak\txSharp}
\newcommand{\atxdSharp}{\nobreak\txdSharp}
\newcommand{\atxNatural}{\nobreak\txNatural}
\newcommand{\btxFlat}{\txFlat\nobreak}
\newcommand{\btxdFlat}{\txdFlat\nobreak}
\newcommand{\btxSharp}{\txSharp\nobreak}
\newcommand{\btxdSharp}{\txdSharp\nobreak}
\newcommand{\btxNatural}{\txNatural\nobreak}

%    \end{macrocode}
% \subsection{\Subscription}
%    \begin{macrocode}
%%==============================================================================
%% 添え字(graphicx,hyperref依存)
%%==============================================================================
%    \end{macrocode}
%    \begin{macrocode}
\newlength{\subsclength}%
\newlength{\supsclength}%
\newlength{\sclength}%
%    \end{macrocode}
% \begin{macro}{\subsc}\marg{text}\par
% コード・シンボルなどの下付き文字を出力します.
% 通常の下付き文字なら\LaTeX カーネルの\cmd{\textsubscript}を
% 使った方が良いんじゃないかと思います.
%    \begin{macrocode}
\DeclareRobustCommand{\subsc}[1]{%
  \texorpdfstring{%TeX
    \nobreak\eghostguarded{\raisebox{-0.1zh}{\scalebox{0.8}{#1}}}%
  }{#1}%
}
%    \end{macrocode}
% \end{macro}
% \begin{macro}{\supsc}\marg{text}\par
% コード・シンボルなどの上付き文字を出力します.
% 通常の上付き文字なら\LaTeX カーネルの\cmd{\textsuperscript}を
% 使った方が良いんじゃないかと思います.
%    \begin{macrocode}
\DeclareRobustCommand{\supsc}[1]{%
  \texorpdfstring{%TeX
    \nobreak\eghostguarded{\raisebox{0.5zh}{\scalebox{0.8}{#1}}}%
  }{#1}%
}

%    \end{macrocode}
% \end{macro}
% \begin{macro}{\subsupsc}\marg{sub-text}\marg{sup-text}\par
% 古典和声記号や芸大和声式の和音記号用の下付き・上付き文字を出力します.
%
% [2020-04-11] 芸大方式の下付き・上付き文字は中央揃えであるよりも左揃えの方が
% 好ましいかも.
% またドイツの機能和声方式では左揃えの方が好ましいかも.
% というわけで,左揃えを標準として組み直しました.
% 右揃えにできるよう星付き亜種を設定しました.
% また単位\texttt{zh}を\cmd{\bell@zh}に変更しました.
%    \begin{macrocode}
\DeclareRobustCommand{\subsupsc}{\@ifstar{\gcmc@ssubsupsc}{\gcmc@subsupsc}}%
\def\gcmc@subsupsc#1#2{%
  \texorpdfstring{%TeX
    \settowidth{\subsclength}{\scalebox{0.7}{#1}}%
    \settowidth{\supsclength}{\scalebox{0.7}{#2}}%
    \ifdim\subsclength<\supsclength\setlength{\sclength}{\supsclength}%
    \else\setlength{\sclength}{\subsclength}\fi%
    \nobreak\eghostguarded{%
      \rlap{\raisebox{-0.2\bell@zh}{\scalebox{0.7}{#1}}}%
      \rlap{\raisebox{0.5\bell@zh}{\scalebox{0.7}{#2}}}%
      \hspace{\sclength}%
    }%
  }{#1#2}%
}% 左揃え
\def\gcmc@ssubsupsc#1#2{%
  \texorpdfstring{%TeX
    \settowidth{\subsclength}{\scalebox{0.7}{#1}}%
    \settowidth{\supsclength}{\scalebox{0.7}{#2}}%
    \ifdim\subsclength<\supsclength\setlength{\sclength}{\supsclength}%
    \else\setlength{\sclength}{\subsclength}\fi%
    \nobreak\eghostguarded{%
      \hspace{\sclength}%
      \llap{\raisebox{-0.2\bell@zh}{\scalebox{0.7}{#1}}}%
      \llap{\raisebox{0.5\bell@zh}{\scalebox{0.7}{#2}}}%
    }%
  }{#1#2}%
}% 右揃え

%    \end{macrocode}
% \end{macro}
% \begin{macro}{\lhsc}\marg{sub-text}\marg{sup-text}\par
% コード・シンボル用の下付き・上付き文字を出力します.
% それぞれのテキストは古典和声記号や芸大和声式の和音記号の数字と違い
% 同じ水平位置から始まる必要が無いので,ただ\cmd{\subsc}と\cmd{\supsc}を
% 並べただけです.
%    \begin{macrocode}
\DeclareRobustCommand{\lhsc}[2]{%
  \subsc{#1}\supsc{#2}%
}% コードネーム記法

%    \end{macrocode}
% \end{macro}
%    \begin{macrocode}
\newlength{\lowsclength}%
\newlength{\midsclength}%
\newlength{\hisclength}%
\newlength{\lmhsclength}%
%    \end{macrocode}
% \begin{macro}{\lowsc}
% \begin{macro}{\midsc}
% \begin{macro}{\hisc}
% \begin{macro}{\lmhsc}
%    \begin{macrocode}
\DeclareRobustCommand{\lowsc}[1]
  {\nobreak\eghostguarded{\raisebox{-0.4zh}{\scalebox{0.6}{#1}}}}
\DeclareRobustCommand{\midsc}[1]
  {\nobreak\eghostguarded{\raisebox{0.2zh}{\scalebox{0.6}{#1}}}}
\DeclareRobustCommand{\hisc}[1]
  {\nobreak\eghostguarded{\raisebox{0.8zh}{\scalebox{0.6}{#1}}}}
\DeclareRobustCommand{\lmhsc}[3]{%
  \texorpdfstring{%TeX
    \settowidth{\lowsclength}{\raisebox{-0.4zh}{\scalebox{0.6}{#1}}}%
    \settowidth{\midsclength}{\raisebox{0.2zh}{\scalebox{0.6}{#2}}}%
    \settowidth{\hisclength}{\raisebox{0.8zh}{\scalebox{0.6}{#3}}}%
    \ifdim\lowsclength<\midsclength\setlength{\lmhsclength}{\midsclength}%
    \else\setlength{\lmhsclength}{\lowsclength}\fi%
    \ifdim\lmhsclength<\hisclength\setlength{\lmhsclength}{\hisclength}\fi%
    \nobreak\eghostguarded{%
      \hspace{0.5\lmhsclength}%\hspace{-0.2em}
      \makebox[0pt][c]{\raisebox{-0.4zh}{\scalebox{0.6}{#1}}}%
      \makebox[0pt][c]{\raisebox{0.2zh}{\scalebox{0.6}{#2}}}%
      \makebox[0pt][c]{\raisebox{0.8zh}{\scalebox{0.6}{#3}}}%
      \hspace{0.5\lmhsclength}%
    }%
  }{#1#2#3}%
}% 三段堆積

%    \end{macrocode}
% \end{macro}
% \end{macro}
% \end{macro}
% \end{macro}
% \subsection{\UsefulMacros}
%    \begin{macrocode}
%%==============================================================================
%% 便利なマクロ(musixtex依存)
%%==============================================================================
%    \end{macrocode}
% \begin{macro}{\setatnextbar}
%    \begin{macrocode}
\newcommand{\setatnextbar}[1]
  {\long\def\atnextbar{#1}}
%    \end{macrocode}
% \end{macro}
% \begin{macro}{\setatnextcenterbar}
%    \begin{macrocode}
\newcommand{\setatnextcenterbar}[1]
  {\long\def\atnextbar{\znotes\centerbar{#1}\en}}

%    \end{macrocode}
% \end{macro}
% \subsection{\ChordSymbols}
%    \begin{macrocode}
%%==============================================================================
%% コードシンボル(graphicx,hyperref依存)
%%==============================================================================
%    \end{macrocode}
% \begin{macro}{\Maj}
% 長七の和音などのための\Maj 記号を出力するコマンドです.
%    \begin{macrocode}
\DeclareRobustCommand{\Maj}
  {\nobreak\textrm{M}}% Major
%    \end{macrocode}
% \end{macro}
% \begin{macro}{\Min}
% 短三和音などのための\Min 記号を出力するコマンドです.
%    \begin{macrocode}
\DeclareRobustCommand{\Min}
  {\nobreak\textrm{m}}% Minor
%    \end{macrocode}
% \end{macro}
% \begin{macro}{\Dim}
% 減七の和音のための\Dim 記号を出力するコマンドです.
%    \begin{macrocode}
\DeclareRobustCommand{\Dim}
  {\nobreak\eghostguarded{\raisebox{-.1zh}{$\circ$}}}% Diminish
%    \end{macrocode}
% \end{macro}
% \begin{macro}{\Dimt}
% 減三和音のための\Dimt 記号を出力するコマンドです.
%    \begin{macrocode}
\DeclareRobustCommand{\Dimt}
  {\nobreak\eghostguarded{\raisebox{.4zh}{$\circ$}}}% Diminish triad
%    \end{macrocode}
% \end{macro}
% \begin{macro}{\hDim}
% 半減七の和音のための\hDim 記号を出力するコマンドです.
%    \begin{macrocode}
\DeclareRobustCommand{\hDim}
  {\nobreak\eghostguarded{\scalebox{0.9}{\o}}}% Half diminish
%    \end{macrocode}
% \end{macro}
% \begin{macro}{\Aug}
% 増三和音のための\Aug 記号を出力するコマンドです.
%    \begin{macrocode}
\DeclareRobustCommand{\Aug}
  {\nobreak\eghostguarded{\raisebox{.4zh}{\scalebox{0.6}{\Plus}}}}
%    \end{macrocode}
% \end{macro}
% \begin{macro}{\Triad}
% 三和音であることを明示するための\Triad 記号を出力するコマンドです.
%    \begin{macrocode}
\DeclareRobustCommand{\Triad}
  {\nobreak\eghostguarded{\subsc\txTri}}
%    \end{macrocode}
% \end{macro}
% \begin{macro}{\Sus}
% Suspendedコードのための\Sus 記号を出力するコマンドです.
%    \begin{macrocode}
\DeclareRobustCommand{\Sus}{%
  \texorpdfstring{% TeX
    \nobreak\eghostguarded{\textrm{sus}}%
  }{sus}% PDF
}% sus

%    \end{macrocode}
% \end{macro}
% \begin{macro}{\Minus}
% マイナス記号\Minus を出力するコマンドです.
%    \begin{macrocode}
\newcommand{\Minus}{%
  \texorpdfstring{% TeX
    \eghostguarded{\raisebox{.1zh}{%
      \ifmmode%
        -%
      \else%
        $-$%
      \fi%
    }}%
  }{-}% PDF
}
%    \end{macrocode}
% \end{macro}
% \begin{macro}{\Plus}
% プラス記号\Plus を出力するコマンドです.
%    \begin{macrocode}
\newcommand{\Plus}{%
  \texorpdfstring{% TeX
    \eghostguarded{\raisebox{.1zh}{%
      \ifmmode%
        +%
      \else%
        $+$%
      \fi%
    }}%
  }{+}% PDF
}
%    \end{macrocode}
% \end{macro}
% \begin{macro}{\Slash}\marg{Slashed text}\par
% 根音省略のための斜線(e.g.~\Slash{V})を
% \marg{Slashed text}に引くためのコマンドです.
%    \begin{macrocode}
\DeclareRobustCommand{\Slash}[1]{%
  \texorpdfstring{%TeX
    \settowidth{\@tempdima}{#1}%
    \settoheight{\@tempdimb}{#1}%
    \settodepth{\@tempdimc}{#1}%
    \addtolength{\@tempdimb}{\@tempdimc}%
    \eghostguarded{%
      \makebox[\@tempdima][c]{%
        \makebox[0pt][l]{%
          \tikz[baseline=\@tempdimc]
            {\draw (0,.4\@tempdimb)--+(\@tempdima,.2\@tempdimb);}%
        }#1%
      }%
    }%
  }{\bell@Slash@onPDF}%
}
%    \end{macrocode}
% \end{macro}
% \begin{macro}{\bell@Slash@onPDF}
% PDFのしおり情報に載せる\cmd{\Slash}の代替テキストです.
%    \begin{macrocode}
\def\bell@Slash@onPDF{根音省略}%
%    \end{macrocode}
% \end{macro}
% \begin{macro}{\chordfrac}\marg{upper text}\marg{lower text}\par
% USTのような,各和音を上下に置く分数コードのシンボルを出力するコマンドです.
%    \begin{macrocode}
\DeclareRobustCommand{\chordfrac}[2]{$\dfrac{\text{#1}}{\text{#2}}$}

%    \end{macrocode}
% \end{macro}
% \begin{macro}{\Mdline}
%    \begin{macrocode}
\newcommand{\Mdline}{\eghostguarded{\rule{1em}{.4pt}}}
%    \end{macrocode}
% \end{macro}
% \begin{macro}{\Muline}
%    \begin{macrocode}
\newcommand{\Muline}{%
  \@tempdima\baselineskip\advance\@tempdima-.4pt\relax
  \eghostguarded{\rule[\@tempdima]{1em}{.4pt}}}
%    \end{macrocode}
% \end{macro}
% \begin{macro}{\Mcline}
%    \begin{macrocode}
\newcommand{\Mcline}{%
  \@tempdima\h@lf\baselineskip\advance\@tempdima-.4pt\relax
  \eghostguarded{\rule[\@tempdima]{1em}{.4pt}}}

%    \end{macrocode}
% \end{macro}
% \subsection{\GeidaiSymbols}
%   \changes{v1.2}{2020/05/04}{芸大和声から移植.}
% 芸大和声の表記法のためのマクロ群です.
%    \begin{macrocode}
%%==============================================================================
%% 和声表記 (hyperref, tikz依存)
%%==============================================================================
%    \end{macrocode}
% \begin{macro}{\onChordsc}\marg{chord}\marg{borrow key}\par
% 借用和音表記(e.g. \onChordsc{V}{V}\subsc7)のためのマクロです.
%    \begin{macrocode}
\DeclareRobustCommand{\onChordsc}[2]{%
  \texorpdfstring{%TeX
    \settowidth{\@tempdima}{#1}%
    \settoheight{\@tempdimb}{#1}%
    \nobreak\eghostguarded{%
      \hspace{0.5\@tempdima}%\hspace{-0.2em}%
      \makebox[0pt][c]{\bgroup#1\egroup}%
      \makebox[0pt][c]{\raisebox{\@tempdimb}{\scalebox{0.5}{\bgroup#2\egroup}}}%
      \hspace{0.5\@tempdima}%
    }%
  }{%PDF
    #2度の#1度%
  }%
}% 借用和音
%    \end{macrocode}
% \end{macro}
% \begin{macro}{\Junko}
% 準固有和音表記のためのマクロです.
% 実態は\Junko を出力するだけのものです.
%    \begin{macrocode}
\DeclareRobustCommand{\Junko}{%
  \texorpdfstring{%TeX
    \ifmmode%
      \nobreak\circ%
    \else%
      \nobreak$\circ$%
    \fi%
  }{%PDF
    o%
  }%
}% 準固有和音

%    \end{macrocode}
% \end{macro}
% \begin{macro}{\bell@ht}
% 変位を含む和音の表記のための\cmd{\dimen}レジスタで,
% 文字の深さと高さを合わせた値を持たせるために使っています.
% 多分テンポラリなレジスタでもいい気がします.
%    \begin{macrocode}
\newdimen\bell@ht
%    \end{macrocode}
% \end{macro}
% \begin{macro}{\uAlt}
% 上方変位された和音(e.g. \uAlt{V})を表記するために使います.
%    \begin{macrocode}
\DeclareRobustCommand{\uAlt}[1]{%
  \texorpdfstring{%TeX
    \tikz[inner sep=0pt,outer sep=0pt]{%
      \node[anchor=west] (T) at (0,0){#1};
      \pgfextracty{\bell@ht}%
                  {\pgfpointdiff{\pgfpointanchor{T}{south}}%
                                {\pgfpointanchor{T}{north}}}%
      \draw (T.north west) -- +(.1em,0.33\bell@ht);
    }%
  }{上方変位#1%
  }%
}% 上方変位(upword Altered)

%    \end{macrocode}
% \end{macro}
% \begin{macro}{\dAlt}
% 下方変位された和音(e.g. \dAlt{V})を表記するために使います.
%    \begin{macrocode}
\DeclareRobustCommand{\dAlt}[1]{%
  \texorpdfstring{%TeX
%    \makebox[0em]{\textsf{'}}\nobreak%
    \tikz[inner sep=0pt,outer sep=0pt]{%
      \node[anchor=west] (T) at (0,0){#1};
      \pgfextracty{\bell@ht}%
                  {\pgfpointdiff{\pgfpointanchor{T}{south}}%
                                {\pgfpointanchor{T}{north}}}%
      \draw (T.north west) -- +(-0.1em,-0.33\bell@ht);
    }%
  }{下方変位#1%
  }%
}% 下方変位(downword Altered)

%    \end{macrocode}
% \end{macro}
% \begin{macro}{\uAltD}
% 上方変位音度の矢印\uAltD を表記します.
%    \begin{macrocode}
\DeclareRobustCommand{\uAltD}{%
  \texorpdfstring{%TeX
    \ifmmode
      \uparrow
    \else
      $\uparrow$%
    \fi
  }{%PDF
    ↑%
  }%
}% 上方変位音度 (upword Altered degree)

%    \end{macrocode}
% \end{macro}
% \begin{macro}{\dAltD}
% 下方変位音度の矢印\dAltD を表記します.
%    \begin{macrocode}
\DeclareRobustCommand{\dAltD}{%
  \texorpdfstring{%TeX
    \ifmmode
      \downarrow
    \else
      $\downarrow$%
    \fi
  }{%PDF
    ↓%
  }%
}% 下方変位音度 (downword Altered degree)
%    \end{macrocode}
% \end{macro}
% \begin{macro}{\txSquare}
% 任意の数字が入る箇所で使われる\txSquare を表記します.
%    \begin{macrocode}
\newcommand{\txSquare}{%
  \texorpdfstring{%TeX
    \ifmmode
      \square
    \else
      $\square$%
    \fi
  }{%PDF
    □%
  }%
}
%    \end{macrocode}
% \end{macro}
% \subsection{\ZCHAR}
%    \begin{macrocode}
%%==============================================================================
%% 楽譜上の文字(graphicx,musixtex依存)
%%==============================================================================
%    \end{macrocode}
% \begin{macro}{\zchar}
%   \changes{v1.1}{2020/02/23}{\file{gachimuchipatch.dtx}から説明を移植.
%                              また定義済みかを確認しないように
%                             (実際これが先に読まれるので).}
% \cmd{\zchar}は次のように定義されています.
% \begin{verbatim}
% \def\zchar#1#2{\check@staff\raise#1\internote\rlap{#2}\fi}
% \end{verbatim}
%
% しかしこの定義では第2引数を受け取る前にカテゴリーコードが確定してしまうため,
% \file{pxrubrica}パッケージでのルビがうまくいきませんでした.
% これを修正します.
% また内部動作を分離することで再利用しやすくします.
%    \begin{macrocode}
\def\zchar{\check@staff\begingroup\endcatcodesmusic\@zchar}
%    \end{macrocode}
% \end{macro}
% \begin{macro}{\@zchar}
% \cmd{\zchar}の内部動作です.
%    \begin{macrocode}
\def\@zchar#1#2{\raise#1\internote\rlap{#2}\endgroup\fi}
%    \end{macrocode}
% \end{macro}
% \begin{macro}{\lchar}
% \cmd{\lchar}も同様に内部動作を分離します.
%    \begin{macrocode}
\def\lchar{\check@staff\begingroup\endcatcodesmusic\@lchar}
%    \end{macrocode}
% \end{macro}
% \begin{macro}{\@lchar}
% \cmd{\lchar}の内部動作です.
%    \begin{macrocode}
\def\@lchar#1#2{\raise#1\internote\llap{#2}\endgroup\fi}
%    \end{macrocode}
% \end{macro}
% \begin{macro}{\cchar}
% \cmd{\cchar}も同様に内部動作を分離します.
%    \begin{macrocode}
\def\cchar{\check@staff\begingroup\endcatcodesmusic\@cchar}
%    \end{macrocode}
% \end{macro}
% \begin{macro}{\@cchar}
% \cmd{\cchar}の内部動作です.
%    \begin{macrocode}
\def\@cchar#1#2{\hroff{\raise#1\internote\lrlap{#2}}\endgroup\fi}

%    \end{macrocode}
% \end{macro}
% \begin{macro}{\lrchar}\marg{pitch}\marg{\&-separated-text}\par\noindent
% タブ記号\verb!&!で分割したテキストに対して,その前後で右詰め・左詰めにした
% テキストを出力します.
% 動作としては\cmd{\lchar}\marg{pitch}\marg{left-text}%
% \cmd{\zchar}\marg{pitch}\marg{right-text}とほぼ同じです
% (ただし\meta{left-text}および\meta{right-text}は
%   それぞれ\verb!&!の左・右のテキスト)
% \changes{v1.1}{2020/02/23}%
%         {Bug fix, remove unnecessary commands.}
%    \begin{macrocode}
\newcommand{\lrchar}{\check@staff\begingroup\endcatcodesmusic\@lrchar}
%    \end{macrocode}
% \end{macro}
% \begin{macro}{\@lrchar}
% \changes{v1.1}{2020/02/23}
%         {Remove unnecessary commands. Change this to be defined by \noexpand\cmd{\def}.}
%    \begin{macrocode}
\def\@lrchar#1#2{%
  \bell@lrchar{#1}#2\af@char
  \endgroup\fi%
}
%    \end{macrocode}
% \end{macro}
% \begin{macro}{\bell@lrchar}
% \changes{v1.1}{2020/02/23}
%         {Change this definition.}
%    \begin{macrocode}
\def\bell@lrchar#1#2&#3\af@char{\lchar{#1}{#2}\zchar{#1}{#3}}

%    \end{macrocode}
% \end{macro}
% \begin{macro}{\zchars}
% \cmd{\zchar}の\cmd{\small}版です.
% 内部で\cmd{@zchar}を読み込んでいます.
%    \begin{macrocode}
\newcommand{\zchars}{\check@staff\begingroup\endcatcodesmusic\small\@zchar}
%    \end{macrocode}
% \end{macro}
% \begin{macro}{\lchars}
% \cmd{\lchar}の\cmd{\small}版です.
% 内部で\cmd{@lchar}を読み込んでいます.
%    \begin{macrocode}
\newcommand{\lchars}{\check@staff\begingroup\endcatcodesmusic\small\@lchar}
%    \end{macrocode}
% \end{macro}
% \begin{macro}{\cchars}
% \cmd{\cchar}の\cmd{\small}版です.
% 内部で\cmd{@cchar}を読み込んでいます.
%    \begin{macrocode}
\newcommand{\cchars}{\check@staff\begingroup\endcatcodesmusic\small\@cchar}
%    \end{macrocode}
% \end{macro}
% \begin{macro}{\lrchars}
% \cmd{\lrchar}の\cmd{\small}版です.
% 内部で\cmd{@lrchar}を読み込んでいます.
%    \begin{macrocode}
\newcommand{\lrchars}{\check@staff\begingroup\endcatcodesmusic\small\@lrchar}

%    \end{macrocode}
% \end{macro}
% \begin{macro}{\gcmc@Factory@char}
%    \begin{macrocode}
\def\gcmc@Factory@char#1{\gcmc@test@temptoka{#1}%
  \csedef{\the\gcmc@test@temptoka charl}%
    {\expandonce{\csname \the\gcmc@test@temptoka chars\endcsname{-8}}}%
  \csedef{\the\gcmc@test@temptoka charll}%
    {\expandonce{\csname \the\gcmc@test@temptoka chars\endcsname{-13}}}%
  \csedef{\the\gcmc@test@temptoka charlll}%
    {\expandonce{\csname \the\gcmc@test@temptoka chars\endcsname{-18}}}%
  \csedef{\the\gcmc@test@temptoka charu}%
    {\expandonce{\csname \the\gcmc@test@temptoka chars\endcsname{12}}}%
  \csedef{\the\gcmc@test@temptoka charuu}%
    {\expandonce{\csname \the\gcmc@test@temptoka chars\endcsname{17}}}%
  \csedef{\the\gcmc@test@temptoka charuuu}%
    {\expandonce{\csname \the\gcmc@test@temptoka chars\endcsname{22}}}%
}

%    \end{macrocode}
%    \begin{macrocode}
\gcmc@Factory@char{z}
\gcmc@Factory@char{l}
\gcmc@Factory@char{c}
\gcmc@Factory@char{lr}

%    \end{macrocode}
% \end{macro}
% \begin{macro}{\gcmc@Factory@chars}
%    \begin{macrocode}
\def\gcmc@Factory@chars#1{\gcmc@test@temptoka{#1}%
  \csedef{\the\gcmc@test@temptoka charsl}%
    {\expandonce{\csname \the\gcmc@test@temptoka chars\endcsname{-8}}}%
  \csedef{\the\gcmc@test@temptoka charsll}%
    {\expandonce{\csname \the\gcmc@test@temptoka chars\endcsname{-13}}}%
  \csedef{\the\gcmc@test@temptoka charslll}%
    {\expandonce{\csname \the\gcmc@test@temptoka chars\endcsname{-18}}}%
  \csedef{\the\gcmc@test@temptoka charsu}%
    {\expandonce{\csname \the\gcmc@test@temptoka chars\endcsname{12}}}%
  \csedef{\the\gcmc@test@temptoka charsuu}%
    {\expandonce{\csname \the\gcmc@test@temptoka chars\endcsname{17}}}%
  \csedef{\the\gcmc@test@temptoka charsuuu}%
    {\expandonce{\csname \the\gcmc@test@temptoka chars\endcsname{22}}}%
}

%    \end{macrocode}
%    \begin{macrocode}
\gcmc@Factory@chars{z}
\gcmc@Factory@chars{l}
\gcmc@Factory@chars{c}
\gcmc@Factory@chars{lr}

%    \end{macrocode}
% \end{macro}
% \begin{macro}{\zcharf}
%    \begin{macrocode}
\newcommand{\zcharf}
  {\check@staff\begingroup\endcatcodesmusic\footnotesize\@zchar}
%    \end{macrocode}
% \end{macro}
% \begin{macro}{\lcharf}
%    \begin{macrocode}
\newcommand{\lcharf}
  {\check@staff\begingroup\endcatcodesmusic\footnotesize\@lchar}
%    \end{macrocode}
% \end{macro}
% \begin{macro}{\ccharf}
%    \begin{macrocode}
\newcommand{\ccharf}
  {\check@staff\begingroup\endcatcodesmusic\footnotesize\@cchar}
%    \end{macrocode}
% \end{macro}
% \begin{macro}{\lrcharf}
%    \begin{macrocode}
\newcommand{\lrcharf}
  {\check@staff\begingroup\endcatcodesmusic\footnotesize\@lrchar}

%    \end{macrocode}
% \end{macro}
% \begin{macro}{\gcmc@Factory@charf}
%    \begin{macrocode}
\def\gcmc@Factory@charf#1{\gcmc@test@temptoka{#1}%
  \csedef{\the\gcmc@test@temptoka charfl}%
    {\expandonce{\csname \the\gcmc@test@temptoka charf\endcsname{-8}}}%
  \csedef{\the\gcmc@test@temptoka charfll}%
    {\expandonce{\csname \the\gcmc@test@temptoka charf\endcsname{-13}}}%
  \csedef{\the\gcmc@test@temptoka charflll}%
    {\expandonce{\csname \the\gcmc@test@temptoka charf\endcsname{-18}}}%
  \csedef{\the\gcmc@test@temptoka charfu}%
    {\expandonce{\csname \the\gcmc@test@temptoka charf\endcsname{12}}}%
  \csedef{\the\gcmc@test@temptoka charfuu}%
    {\expandonce{\csname \the\gcmc@test@temptoka charf\endcsname{17}}}%
  \csedef{\the\gcmc@test@temptoka charfuuu}%
    {\expandonce{\csname \the\gcmc@test@temptoka charf\endcsname{22}}}%
}

%    \end{macrocode}
% \end{macro}
%    \begin{macrocode}
\gcmc@Factory@charf{z}
\gcmc@Factory@charf{l}
\gcmc@Factory@charf{c}
\gcmc@Factory@charf{lr}

%    \end{macrocode}
% \begin{macro}{\zcharsc}
%    \begin{macrocode}
\newcommand{\zcharsc}
  {\check@staff\begingroup\endcatcodesmusic\scriptsize\@zchar}
%    \end{macrocode}
% \end{macro}
% \begin{macro}{\lcharsc}
%    \begin{macrocode}
\newcommand{\lcharsc}
  {\check@staff\begingroup\endcatcodesmusic\scriptsize\@lchar}
%    \end{macrocode}
% \end{macro}
% \begin{macro}{\ccharsc}
%    \begin{macrocode}
\newcommand{\ccharsc}
  {\check@staff\begingroup\endcatcodesmusic\scriptsize\@cchar}
%    \end{macrocode}
% \end{macro}
% \begin{macro}{\lrcharsc}
%    \begin{macrocode}
\newcommand{\lrcharsc}
  {\check@staff\begingroup\endcatcodesmusic\scriptsize\@lrchar}

%    \end{macrocode}
% \end{macro}
% \begin{macro}{\gcmc@Factory@charsc}
%    \begin{macrocode}
\def\gcmc@Factory@charsc#1{\gcmc@test@temptoka{#1}%
  \csedef{\the\gcmc@test@temptoka charscl}%
    {\expandonce{\csname \the\gcmc@test@temptoka charsc\endcsname{-8}}}%
  \csedef{\the\gcmc@test@temptoka charscll}%
    {\expandonce{\csname \the\gcmc@test@temptoka charsc\endcsname{-13}}}%
  \csedef{\the\gcmc@test@temptoka charsclll}%
    {\expandonce{\csname \the\gcmc@test@temptoka charsc\endcsname{-18}}}%
  \csedef{\the\gcmc@test@temptoka charscu}%
    {\expandonce{\csname \the\gcmc@test@temptoka charsc\endcsname{12}}}%
  \csedef{\the\gcmc@test@temptoka charscuu}%
    {\expandonce{\csname \the\gcmc@test@temptoka charsc\endcsname{17}}}%
  \csedef{\the\gcmc@test@temptoka charscuuu}%
    {\expandonce{\csname \the\gcmc@test@temptoka charsc\endcsname{22}}}%
}

%    \end{macrocode}
% \end{macro}
%    \begin{macrocode}
\gcmc@Factory@charsc{z}
\gcmc@Factory@charsc{l}
\gcmc@Factory@charsc{c}
\gcmc@Factory@charsc{lr}

%    \end{macrocode}
% \begin{macro}{\Startpiece}
%    \begin{macrocode}
\newcommand{\Startpiece}{\startpiece\addspace\afterruleskip}

%    \end{macrocode}
% \end{macro}
% \begin{macro}{\octnumberup}
%    \begin{macrocode}
\def\octnumberup{%
  \octnumberdefault\scalebox{0.8}{\txtfont va}\vrule width 0pt height2ex}
%    \end{macrocode}
% \end{macro}
% \begin{macro}{\octnumberdefault}
%    \begin{macrocode}
\def\octnumberdefault{%
  \smallppff8\nobreak}
%    \end{macrocode}
% \end{macro}
% \begin{macro}{\octnumberdown}
%    \begin{macrocode}
\def\octnumberdown{%
  \octnumberdefault\scalebox{0.8}{\txtfont vb}\vrule width 0pt depth \Internote}

%    \end{macrocode}
% \end{macro}
% \subsection{\ZCHORD}
%    \begin{macrocode}
%%==============================================================================
%% 和声記号配置(musixtex依存)
%%==============================================================================
%    \end{macrocode}
%    \begin{macrocode}
\newcommand{\zchord}{\check@staff\begingroup\endcatcodesmusic\@zchord}
\newcommand{\lchord}{\check@staff\begingroup\endcatcodesmusic\@lchord}
\newcommand{\cchord}{\cchar}
\newcommand{\lrchord}{%
  \check@staff\begingroup\endcatcodesmusic\catcode\`&=4\@lrchord}

%    \end{macrocode}
%    \begin{macrocode}
\def\gcmc@Factory@chord#1{\gcmc@test@temptoka{#1}%
  \csedef{\the\gcmc@test@temptoka chordl}%
    {\expandonce{\csname \the\gcmc@test@temptoka chords\endcsname{-8}}}%
  \csedef{\the\gcmc@test@temptoka chordll}%
    {\expandonce{\csname \the\gcmc@test@temptoka chords\endcsname{-13}}}%
  \csedef{\the\gcmc@test@temptoka chordlll}%
    {\expandonce{\csname \the\gcmc@test@temptoka chords\endcsname{-18}}}%
  \csedef{\the\gcmc@test@temptoka chordu}%
    {\expandonce{\csname \the\gcmc@test@temptoka chords\endcsname{12}}}%
  \csedef{\the\gcmc@test@temptoka chorduu}%
    {\expandonce{\csname \the\gcmc@test@temptoka chords\endcsname{17}}}%
  \csedef{\the\gcmc@test@temptoka chorduuu}%
    {\expandonce{\csname \the\gcmc@test@temptoka chords\endcsname{22}}}%
}

%    \end{macrocode}
%    \begin{macrocode}
\gcmc@Factory@chord{z}
\gcmc@Factory@chord{l}
\gcmc@Factory@chord{c}
\gcmc@Factory@chord{lr}

%    \end{macrocode}
%    \begin{macrocode}
\newcommand{\zchords}{\check@staff\begingroup\endcatcodesmusic\small\@zchord}
\newcommand{\lchords}{\check@staff\begingroup\endcatcodesmusic\small\@lchord}
\newcommand{\cchords}{\check@staff\begingroup\endcatcodesmusic\small\@cchar}
\newcommand{\lrchords}{\check@staff\begingroup\endcatcodesmusic\small\@lrchord}

%    \end{macrocode}
%    \begin{macrocode}
\def\gcmc@Factory@chords#1{\gcmc@test@temptoka{#1}%
  \csedef{\the\gcmc@test@temptoka chordsl}%
    {\expandonce{\csname \the\gcmc@test@temptoka chords\endcsname{-8}}}%
  \csedef{\the\gcmc@test@temptoka chordsll}%
    {\expandonce{\csname \the\gcmc@test@temptoka chords\endcsname{-13}}}%
  \csedef{\the\gcmc@test@temptoka chordslll}%
    {\expandonce{\csname \the\gcmc@test@temptoka chords\endcsname{-18}}}%
  \csedef{\the\gcmc@test@temptoka chordsu}%
    {\expandonce{\csname \the\gcmc@test@temptoka chords\endcsname{12}}}%
  \csedef{\the\gcmc@test@temptoka chordsuu}%
    {\expandonce{\csname \the\gcmc@test@temptoka chords\endcsname{17}}}%
  \csedef{\the\gcmc@test@temptoka chordsuuu}%
    {\expandonce{\csname \the\gcmc@test@temptoka chords\endcsname{22}}}%
}

%    \end{macrocode}
%    \begin{macrocode}
\gcmc@Factory@chords{z}
\gcmc@Factory@chords{l}
\gcmc@Factory@chords{c}
\gcmc@Factory@chords{lr}

%    \end{macrocode}
%    \begin{macrocode}
\newcommand{\zchordf}
  {\check@staff\begingroup\endcatcodesmusic\footnotesize\@zchord}
\newcommand{\lchordf}
  {\check@staff\begingroup\endcatcodesmusic\footnotesize\@lchord}
\newcommand{\cchordf}
  {\check@staff\begingroup\endcatcodesmusic\footnotesize\@cchar}
\newcommand{\lrchordf}
  {\check@staff\begingroup\endcatcodesmusic\footnotesize\@lrchord}

%    \end{macrocode}
%    \begin{macrocode}
\def\gcmc@Factory@chordf#1{\gcmc@test@temptoka{#1}%
  \csedef{\the\gcmc@test@temptoka chordfl}%
    {\expandonce{\csname \the\gcmc@test@temptoka chordf\endcsname{-8}}}%
  \csedef{\the\gcmc@test@temptoka chordfll}%
    {\expandonce{\csname \the\gcmc@test@temptoka chordf\endcsname{-13}}}%
  \csedef{\the\gcmc@test@temptoka chordflll}%
    {\expandonce{\csname \the\gcmc@test@temptoka chordf\endcsname{-18}}}%
  \csedef{\the\gcmc@test@temptoka chordfu}%
    {\expandonce{\csname \the\gcmc@test@temptoka chordf\endcsname{12}}}%
  \csedef{\the\gcmc@test@temptoka chordfuu}%
    {\expandonce{\csname \the\gcmc@test@temptoka chordf\endcsname{17}}}%
  \csedef{\the\gcmc@test@temptoka chordfuuu}%
    {\expandonce{\csname \the\gcmc@test@temptoka chordf\endcsname{22}}}%
}

%    \end{macrocode}
%    \begin{macrocode}
\gcmc@Factory@chordf{z}
\gcmc@Factory@chordf{l}
\gcmc@Factory@chordf{c}
\gcmc@Factory@chordf{lr}

%    \end{macrocode}
%    \begin{macrocode}
\newcommand{\zchordsc}
  {\check@staff\begingroup\endcatcodesmusic\scriptsize\@zchord}
\newcommand{\lchordsc}
  {\check@staff\begingroup\endcatcodesmusic\scriptsize\@lchord}
\newcommand{\cchordsc}
  {\check@staff\begingroup\endcatcodesmusic\scriptsize\@cchar}
\newcommand{\lrchordsc}
  {\check@staff\begingroup\endcatcodesmusic\scriptsize\@lrchord}

%    \end{macrocode}
%    \begin{macrocode}
\def\gcmc@Factory@chordsc#1{\gcmc@test@temptoka{#1}%
  \csedef{\the\gcmc@test@temptoka chordscl}%
    {\expandonce{\csname \the\gcmc@test@temptoka chordsc\endcsname{-8}}}%
  \csedef{\the\gcmc@test@temptoka chordscll}%
    {\expandonce{\csname \the\gcmc@test@temptoka chordsc\endcsname{-13}}}%
  \csedef{\the\gcmc@test@temptoka chordsclll}%
    {\expandonce{\csname \the\gcmc@test@temptoka chordsc\endcsname{-18}}}%
  \csedef{\the\gcmc@test@temptoka chordscu}%
    {\expandonce{\csname \the\gcmc@test@temptoka chordsc\endcsname{12}}}%
  \csedef{\the\gcmc@test@temptoka chordscuu}%
    {\expandonce{\csname \the\gcmc@test@temptoka chordsc\endcsname{17}}}%
  \csedef{\the\gcmc@test@temptoka chordscuuu}%
    {\expandonce{\csname \the\gcmc@test@temptoka chordsc\endcsname{22}}}%
}

%    \end{macrocode}
%    \begin{macrocode}
\gcmc@Factory@chordsc{z}
\gcmc@Factory@chordsc{l}
\gcmc@Factory@chordsc{c}
\gcmc@Factory@chordsc{lr}

%    \end{macrocode}
%    \begin{macrocode}
\newcommand{\@lrchord}[2]{%
  \def\reserved@a{#1}\def\reserved@b{#2}%
  \expandafter\bell@lrchord\reserved@b\af@chord
  \endgroup\fi%
}

%    \end{macrocode}
%    \begin{macrocode}
\newcommand{\@zchord}[2]{\@zchar{#1}{\hspace{-.1\wn@width}#2}}
\newcommand{\@lchord}[2]{\@lchar{#1}{#2\hspace{.1\wn@width}}}
\def\bell@lrchord#1&#2\af@chord{%
  \lchord{\reserved@a}{#1}\zchord{\reserved@a}{#2}}

%    \end{macrocode}
% \subsection{\FigureBass}
%    \begin{macrocode}
%%==============================================================================
%% 数字付き低音(graphicx依存)
%%==============================================================================
%    \end{macrocode}
%    \begin{macrocode}
\newcount\figbass@cnta% 低音数字の配列のイテレータ
\newdimen\figbass@dima%
\newdimen\figbass@dimb%
\newdimen\figbass@dimc%
\def\figbass@coefficientOfSize{0.65}% 数字の大きさのスケーリング
\figbass@dimb2.3\internote % 数字同士の縦の間隔
%    \end{macrocode}
% \begin{macro}{\figbass}
% \cmd{\figbass}\oarg{figure-number 1}\oarg{figure-number 2}$\ldots$%
%               \oarg{figure-number n}\marg{height}の形で
% 数字付き低音を表示します.
% \oarg{figure-number n}は下から$n$番目の数字です.
% \marg{height}は表示する高さをinternote単位で指定します.
% このとき正であれば指定した高さから上へ積み重ね,
% 負であれば指定した高さが最も上になるよう下から上へ積み上げます.
%    \begin{macrocode}
%% \figbass[#1][#2]...[#n-1]{#n}
%% #1,...,#n-1: figure number
%% #n: height of figure number [internote]
\newcommand{\figbass}{\figbass@dima\z@\bell@figbass}
%    \end{macrocode}
% \end{macro}
%    \begin{macrocode}
\newcommand{\figbassu}{\figbass@dima8.5\internote\bell@figbass}
\newcommand{\figbassuu}{\figbass@dima10.5\internote\bell@figbass}
\newcommand{\figbassuuu}{\figbass@dima12.5\internote\bell@figbass}
\newcommand{\figbassuuuu}{\figbass@dima14.5\internote\bell@figbass}
\newcommand{\figbassl}{\figbass@dima-2.5\internote\bell@figbass}
\newcommand{\figbassll}{\figbass@dima-4.5\internote\bell@figbass}
\newcommand{\figbasslll}{\figbass@dima-6.5\internote\bell@figbass}
%    \end{macrocode}
% \begin{macro}{\bell@figbass}
% カテゴリーコードを通常に戻して内部処理を進めます.
%    \begin{macrocode}
\def\bell@figbass{\check@staff\begingroup\endcatcodesmusic\bell@figb@ss}
%    \end{macrocode}
% \end{macro}
% \begin{macro}{\bell@figb@ss}
% 数字が残っているかどうかで処理を分けます.
% 残っていれば\cmd{\@figbass@}でそれを処理し,残っていない場合は
% 高さを指定する(\cmd{\@figbass@nocnt@})か,
% あるいは既に指定されている場合はそれを使用します(\cmd{\@figbass@cnt@}).
%    \begin{macrocode}
\def\bell@figb@ss{%
  \@ifnextchar[{%
    \@figbass@% 低音数字の配列
  }{%
    \ifnum\figbass@dima=\z@
      \expandafter\expandafter\expandafter\@figbass@nocnt@% 場所指定の数字付き
    \else
      \expandafter\expandafter\expandafter\@figbass@cnt@% 数字付き
    \fi\fi
%   \ifnum\figbass@dima>\z@
%     \let\@figb@ss\@figbass@cnt@% 上に数字付き
%   \else\ifnum\figbass@dima<\z@
%     \let\@figb@ss\@figbass@cnt@% 下に数字付き
%   \else
%     \let\@figb@ss\@figbass@nocnt@% 場所指定の数字付き
%   \fi\fi\fi\@figb@ss
  }%
}
%    \end{macrocode}
% \end{macro}
% \begin{macro}{\@figbass@}
% 疑似配列マクロ\cmd{\@figb@ss<roman-number>}に数字を定義します.
% 出力時にこれは参照されます.
%    \begin{macrocode}
\def\@figbass@[#1]{%
  \advance\figbass@cnta\@ne
  \csdef{@figb@ss@\romannumeral\figbass@cnta}{\normalsize #1}%
  \bell@figb@ss
}
%    \end{macrocode}
% \end{macro}
% \begin{macro}{\@figbass@cnt@}
% 単純に\cmd{\@figbass@@}を呼び出して数字を出力します.
%    \begin{macrocode}
\def\@figbass@cnt@{\@figbass@@}% 上(下)に数字付き
%    \end{macrocode}
% \end{macro}
% \begin{macro}{\@figbass@nocnt@}\marg{height}\par
% \cmd{\figbass@dima}に\meta{height}を代入してから
% \cmd{\@figbass@@}を呼び出します.
%    \begin{macrocode}
\def\@figbass@nocnt@#1{\figbass@dima#1\internote\@figbass@@}% 場所指定の数字付き
%    \end{macrocode}
% \end{macro}
% \begin{macro}{\@figbass@@}
% 実際に数字付き低音の出力をします.
% 文字の縮小は\cmd{\scalebox}で行っているので\file{graphicx}パッケージ依存です.
%    \begin{macrocode}
\def\@figbass@@{% 数字付き低音の出力
  \figbass@dimc\figbass@dimb \multiply\figbass@dimc\figbass@cnta
  \ifnum\figbass@dima>\z@ \advance\figbass@dima\figbass@dimc \fi
  \loop\ifnum\figbass@cnta>0\relax
    \advance\figbass@dima-\figbass@dimb
    \@figbass@char{\figbass@dima}
                  {\scalebox
                    {\figbass@coefficientOfSize}
                    {\csname @figb@ss@\romannumeral\figbass@cnta\endcsname}%
                  }% 0.540727 = 2\internote/ 1zh
    \advance\figbass@cnta\m@ne
  \repeat
  \endgroup\figbass@dima\z@
}
%    \end{macrocode}
% \end{macro}
% \begin{macro}{\@figbass@char}
% 出力される数字を右揃えにするためのマクロです.
%    \begin{macrocode}
\def\@figbass@char#1#2{\roffset{0.8}{\raise#1\llap{#2}}}

%    \end{macrocode}
% \end{macro}
% \subsection{\Tenni}
%    \begin{macrocode}
%%==============================================================================
%% 転位記号
%%==============================================================================
%    \end{macrocode}
%    \begin{macrocode}
\newdimen\gcmc@dPos@skip
\newdimen\gcmc@Posparaboraheighta
\newdimen\gcmc@Posparaboraheightb
\def\gcmc@Posparabora@coeff{.5ex}
\gcmc@dPos@skip=1pt

\DeclareRobustCommand{\dPos}[1]{%
  \settowidth{\@tempdima}{#1}%
  \eghostguarded{%
    \hbox\@to\@tempdima{\rlap{\tikz[baseline=\gcmc@dPos@skip]{%
      \pgfmathsetlength{\gcmc@Posparaboraheighta}{\gcmc@Posparabora@coeff}%
      \draw (0,0)
        parabola[parabola height=-\gcmc@Posparaboraheighta]
        +(\@tempdima,0);}}#1}}}

\DeclareRobustCommand{\uPos}[1]{%
  \settowidth{\@tempdima}{#1}%
  \settoheight{\@tempdimb}{#1}%
  \eghostguarded{%
    \advance\@tempdimb\gcmc@dPos@skip\relax
    \hbox\@to\@tempdima{\rlap{\tikz[baseline=-\@tempdimb]{%
      \pgfmathsetlength{\gcmc@Posparaboraheighta}{\gcmc@Posparabora@coeff}%
      \draw (0,0)
        parabola[parabola height=\gcmc@Posparaboraheighta]
        +(\@tempdima,0);}}#1}}}

\DeclareRobustCommand{\ddPos}[1]{%
  \settowidth{\@tempdima}{#1}%
  \eghostguarded{%
    \hbox\@to\@tempdima{\rlap{\tikz[baseline=\gcmc@dPos@skip]{%
      \pgfmathsetlength{\gcmc@Posparaboraheighta}{\gcmc@Posparabora@coeff}%
      \pgfmathsetlength{\gcmc@Posparaboraheightb}{\gcmc@Posparaboraheighta*3/5}%
      \draw (0,0) parabola[parabola height=-\gcmc@Posparaboraheighta]
            +(\@tempdima,0);
      \draw (.2\@tempdima,0) parabola[parabola height=-\gcmc@Posparaboraheightb]
            +(.6\@tempdima,0);}}#1}}}

\DeclareRobustCommand{\duPos}[1]{%
  \settowidth{\@tempdima}{#1}%
  \eghostguarded{%
    \hbox\@to\@tempdima{\rlap{\tikz[baseline=\gcmc@dPos@skip]{%
      \pgfmathsetlength{\gcmc@Posparaboraheighta}%
                       {\gcmc@Posparabora@coeff}%
      \pgfmathsetlength{\gcmc@Posparaboraheightb}%
                       {\gcmc@Posparaboraheighta*16/25}% .333*16/25
      \draw (0,0)
        parabola[parabola height=-\gcmc@Posparaboraheighta] +(\@tempdima,0);
      \draw ($(.2\@tempdima,-\gcmc@Posparaboraheighta)
              -(0, -\gcmc@Posparaboraheightb)$)
              parabola[parabola height=\gcmc@Posparaboraheightb]
              +(.6\@tempdima,0);}}#1}}}

\DeclareRobustCommand{\uuPos}[1]{%
  \settowidth{\@tempdima}{#1}%
  \settoheight{\@tempdimb}{#1}%
  \eghostguarded{%
    \advance\@tempdimb\gcmc@dPos@skip\relax
    \hbox\@to\@tempdima{\rlap{\tikz[baseline=-\@tempdimb]{%
      \pgfmathsetlength{\gcmc@Posparaboraheighta}%
                       {\gcmc@Posparabora@coeff}%
      \pgfmathsetlength{\gcmc@Posparaboraheightb}%
                       {\gcmc@Posparaboraheighta*3/5}% .32*3/5
      \draw (0,0) parabola[parabola height=\gcmc@Posparaboraheighta]
              +(\@tempdima,0);
      \draw (.2\@tempdima,0) parabola[parabola height=\gcmc@Posparaboraheightb]
              +(.6\@tempdima,0);}}#1}}}

\DeclareRobustCommand{\udPos}[1]{%
  \settowidth{\@tempdima}{#1}%
  \settoheight{\@tempdimb}{#1}%
  \eghostguarded{%
    \advance\@tempdimb\gcmc@dPos@skip\relax
    \hbox\@to\@tempdima{\rlap{\tikz[baseline=-\@tempdimb]{%
      \pgfmathsetlength{\gcmc@Posparaboraheighta}%
                       {\gcmc@Posparabora@coeff}%
      \pgfmathsetlength{\gcmc@Posparaboraheightb}%
                       {\gcmc@Posparaboraheighta*16/25}%
      \draw (0,0) parabola[parabola height=\gcmc@Posparaboraheighta]
              +(\@tempdima,0);
      \draw ($(.2\@tempdima,\gcmc@Posparaboraheighta)
               - (0, \gcmc@Posparaboraheightb)$)
            parabola[parabola height=-\gcmc@Posparaboraheightb]
              +(.6\@tempdima,0);}}#1}}}

\DeclareRobustCommand{\dPosdAlt}[1]{%
  \settowidth{\@tempdima}{#1}%
  \eghostguarded{%
    \hbox\@to\@tempdima{\rlap{\tikz[baseline=\gcmc@dPos@skip]{%
      \pgfmathsetlength{\gcmc@Posparaboraheighta}%
                       {\gcmc@Posparabora@coeff}%
      \draw (0,0)
              parabola[parabola height=-\gcmc@Posparaboraheighta]
              +(\@tempdima,0);
      \draw[arrows ={-Stealth[length=.3ex,width=.25em,inset=.15ex]}]
            (.5\@tempdima,0) -- +(0,-\gcmc@Posparaboraheighta);}}#1}}}

\DeclareRobustCommand{\dPosuAlt}[1]{%
  \settowidth{\@tempdima}{#1}%
  \eghostguarded{%
    \hbox\@to\@tempdima{\rlap{\tikz[baseline=\gcmc@dPos@skip]{%
      \pgfmathsetlength{\gcmc@Posparaboraheighta}%
                       {\gcmc@Posparabora@coeff}%
      \draw (0,0)
              parabola[parabola height=-\gcmc@Posparaboraheighta]
              +(\@tempdima,0);
      \draw[arrows ={-Stealth[length=.3ex,width=.25em,inset=.15ex]}]
            (.5\@tempdima,-\gcmc@Posparaboraheighta)
              -- +(0,\gcmc@Posparaboraheighta);}}#1}}}
\DeclareRobustCommand{\uPosdAlt}[1]{%
  \settowidth{\@tempdima}{#1}%
  \settoheight{\@tempdimb}{#1}%
  \eghostguarded{%
    \advance\@tempdimb\gcmc@dPos@skip\relax
    \hbox\@to\@tempdima{\rlap{\tikz[baseline=-\@tempdimb]{%
      \pgfmathsetlength{\gcmc@Posparaboraheighta}%
                       {\gcmc@Posparabora@coeff}%
      \draw (0,0) parabola[parabola height=\gcmc@Posparaboraheighta]
              +(\@tempdima,0);
      \draw[arrows ={-Stealth[length=.3ex,width=.25em,inset=.15ex]}]
            (.5\@tempdima,\gcmc@Posparaboraheighta)
              -- +(0,-\gcmc@Posparaboraheighta);}}#1}}}

\DeclareRobustCommand{\uPosuAlt}[1]{%
  \settowidth{\@tempdima}{#1}%
  \settoheight{\@tempdimb}{#1}%
  \eghostguarded{%
    \advance\@tempdimb\gcmc@dPos@skip\relax
    \hbox\@to\@tempdima{\rlap{\tikz[baseline=-\@tempdimb]{%
      \pgfmathsetlength{\gcmc@Posparaboraheighta}%
                       {\gcmc@Posparabora@coeff}%
      \draw (0,0) parabola[parabola height=\gcmc@Posparaboraheighta]
            +(\@tempdima,0);
      \draw[arrows ={-Stealth[length=.3ex,width=.25em,inset=.15ex]}]
            (.5\@tempdima,0)
              -- +(0,\gcmc@Posparaboraheighta);}}#1}}}
%    \end{macrocode}
% \subsection{\Harmonie}
%    \begin{macrocode}
%%==============================================================================
%% ドイツ式和声記号
%%==============================================================================
%    \end{macrocode}
% \begin{macro}{\Funktion}
% \cmd{\Funktion}\marg{funktion-name}の形で
% 主にドイツにて使われる機能和声表示を用いるためのコマンドです.
% \TODO
% いくつかのエイリアスを実装して使いやすくしたいです.
% \par [2020/07/28]
% \cmd{DeclareRobustCommand}で定義しているけど動く引数でうまく動かないです.
% \pkg{etoolbox}の\cmd{\robustify}を使うと上手くいくのですが…….
%    \begin{macrocode}
\DeclareRobustCommand{\Funktion}{\begingroup\bell@Funktion}%
%    \end{macrocode}
% \end{macro}
% \begin{macro}{\bell@Funktion}
%    \begin{macrocode}
\def\bell@Funktion#1{\bell@Funktionstyle#1\endgroup}%
%    \end{macrocode}
% \end{macro}
% \begin{macro}{\bell@Funktionstyle}
%    \begin{macrocode}
\def\bell@Funktionstyle{}%

%    \end{macrocode}
% \end{macro}
% \begin{macro}{\WechselD}
% Doppeldominanteです.
%    \begin{macrocode}
\def\WechselD{%
  \settowidth{\@tempdima}{D}\raisebox{.5ex}{D}\hspace{-\thr@@qu@rt\@tempdima}D%
}% Wechseldominante
%    \end{macrocode}
% \end{macro}
% \begin{macro}{\WechselS}
% Doppelsubdominanteです.
%    \begin{macrocode}
\def\WechselS{%
  \settowidth{\@tempdima}{S}\raisebox{.5ex}{S}\hspace{-\smallvalue\@tempdima}S%
}% Wechselsubdominante
%    \end{macrocode}
% \end{macro}
% \begin{macro}{\Wechsels}
% Doppelsubdominanteです.
%    \begin{macrocode}
\def\Wechsels{%
  \settowidth{\@tempdima}{s}\raisebox{1ex}{ss}%
}% Wechselsubdominante
%    \end{macrocode}
% \end{macro}
% \begin{macro}{\DoppelfunktionDs}
% \changes{v1.3}{2020/11/06}
%   {NEW: \cmd{\DoppelfunktionDs}\DoppelfunktionDs の新設}
% 下属短三和音と上属長三和音の混合によって得られる減七の和音の表記です.
%    \begin{macrocode}
\def\DoppelfunktionDs{%
  \settowidth{\@tempdima}{D}D\hspace{-\qu@rt\@tempdima}\raisebox{.5ex}{s}\hochgestellt{v}%
}% Wechseldominante
%    \end{macrocode}
% \end{macro}
% \begin{macro}{\DoppelfunktionDS}
% \changes{v1.3}{2020/11/06}
%   {NEW: \cmd{\DoppelfunktionDS}\DoppelfunktionDS の新設}
% 下属長三和音と上属長三和音の混合によって得られる半減七の和音の表記です.
%    \begin{macrocode}
\def\DoppelfunktionDS{%
  \settowidth{\@tempdima}{D}D\hspace{-\qu@rt\@tempdima}\raisebox{.5ex}{S}\hochgestellt{v}%
}% Wechseldominante
%    \end{macrocode}
% \end{macro}
% \begin{macro}{\WechselDOhne}
% \changes{v1.3}{2020/11/06}
%   {NEW: \cmd{\WechselDOhne}\WechselDOhne の新設}
% トニカ短三和音とDoppeldominanteの混合によって得られる減七の和音の表記です.
%    \begin{macrocode}
\def\WechselDOhne{%
  \raisebox{.5ex}{t}\WechselD\hochgestellt{v}%
}% Wechseldominante
%    \end{macrocode}
% \end{macro}
% \begin{macro}{\tiefalt}
% Tiefalteration (chromatische Erniedrigung)です.
%    \begin{macrocode}
\def\tiefalt{\eghostguarded{\scalebox{.5}[1]{$>$}}}%変変化音 Tiefalteration
%    \end{macrocode}
% \end{macro}
% \begin{macro}{\hochalt}
% Hochalteration (chromatische Erhöhung)です.
%    \begin{macrocode}
\def\hochalt{\eghostguarded{\scalebox{.5}[1]{$<$}}}%嬰変化音 Hochalteration
%    \end{macrocode}
% \end{macro}
% \begin{macro}{\tiefgestellt}
% \changes{v1.2}{2020/07/11}
%   {下添え字のコマンドを\cmd{\Stellung}から\cmd{\tiefgestellt}に変更.
%    \cmd{\Stellung}は後方互換のために残す.}
% \changes{v1.3}{2020/07/11}
%   {和音表記のdepthを考慮する形に変更.}
% \marg{chord}\marg{bass}の形で低音を指定した和音表記をします.
%    \begin{macrocode}
\def\tiefgestellt#1#2{%
  \eghostguarded{%
    \settowidth{\@tempdima}{#1}%
    \makebox[\@tempdima][c]{%
      \rlap{%
        \settowidth{\@tempdima}{#1}%
        \settowidth{\@tempdimb}{\scalebox{.7}{#2}}%
        \advance\@tempdima-\@tempdimb
        \hspace{\h@lf\@tempdima}%
        \settodepth{\@tempdima}{#1}%
        \settoheight{\@tempdimb}{\scalebox{.7}{#2}}%
        \advance\@tempdima+1\qu@rt\@tempdimb
        \raisebox{-\@tempdima}{\scalebox{.7}{#2}}%
      }#1%
    }%
  }%
}% バス音指定 Die Stellung
%    \end{macrocode}
% \end{macro}
% \begin{macro}{\Stellung}
% \changes{v1.2}{2020/07/11}
%   {下添え字のコマンドを\cmd{\Stellung}から\cmd{\tiefgestellt}に変更.
%    \cmd{\Stellung}は後方互換のために残す.}
% \marg{chord}\marg{bass}は後方互換のために残します.
%    \begin{macrocode}
\let\Stellung\tiefgestellt
%    \end{macrocode}
% \end{macro}
% \begin{macro}{\hochgestellt}
% \changes{v1.2}{2020/07/12}
%   {上添え字のコマンドを\LaTeX の流儀に合わせて\cmd{\Hochgestellt}から
%    \cmd{\hochgestellt}に変更.
%    \cmd{\Hochgestellt}は後方互換のために残す.}
% 右上の添え字を表記します.
%    \begin{macrocode}
\def\hochgestellt#1{\supsc{#1}}% 右上添え字 

%    \end{macrocode}
% \end{macro}
% \begin{macro}{\Hochgestellt}
% \changes{v1.2}{2020/07/12}
%   {上添え字のコマンドを\LaTeX の流儀に合わせて\cmd{\Hochgestellt}から
%    \cmd{\hochgestellt}に変更.
%    \cmd{\Hochgestellt}は後方互換のために残す.}
% \cmd{\Hochgestellt}は後方互換のために残します.
%    \begin{macrocode}
\let\Hochgestellt\hochgestellt% 右上添え字 

%    \end{macrocode}
% \end{macro}
% \subsection{\ChordProg}
% \paragraph{概要}
%    \begin{macrocode}
%%==============================================================================
%% Chord Progressionの表記のため
%%==============================================================================
%    \end{macrocode}
% \begin{macro}{\CProg}
% コード進行を表記するマクロです.
% \cmd{\CProg}\marg{arg}を基本形としますが,
% \cmd{\CProg*}, \cmd{\CProg-}, \cmd{\CProg*-}の形もあります.
% それぞれは次のように使い分けます.
% \begin{itemize}
% \item \cmd{\CProg}は長調に対して使います. 音度記号も同時に出力します.
% \item \cmd{\CProg*}は短調に対して使います. 音度記号も同時に出力します.
% \item \cmd{\CProg-}は音度記号を出力しない形です.
% \item \cmd{\CProg*-}は\cmd{\CProg-}と同じ?
% \end{itemize}
% 
% \meta{arg}は次の方法で記述されます.
% \begin{itemize}
% \item 音度記号が必要なら,\meta{degree\_Symbol}|:|\meta{chord\_Symbol}|;|を
%       連結させていきます.
% \item 音度記号が不要な場合は\meta{chord\_Symbol}|;|を連結させていきます.
% \end{itemize}
%
% \begin{macro}{\CProgl}
% 同様の記述方法で,インラインで使うための|\CProgl|が使用可能です.
% \end{macro}
% \paragraph{実装}
%    \begin{macrocode}
\newif\if@vCProg
\newif\if@lCProg
\@lCProgfalse
% 多分トークンレジスタ使った方が素直な実装になる気がする.
\def\CProg{\@ifstar{\def\tmp@CP@tokA{Minor: }\@CProg}%
                   {\def\tmp@CP@tokA{Major: }\@CProg}}
\def\CProgl{\@lCProgtrue\CProg}
\def\@CProg{\@ifnextchar-{\let\tmp@CP@tokA\relax\@vCProgtrue\@@CProg}%
                         {\@vCProgfalse\@@CProg-}}
\def\@@CProg#1#2{%
  \begingroup
    \if@lCProg\else \par\centering \fi
    \tmp@CP@tokA\h@CProg#2;\relax;%
    \if@lCProg\else \par \fi
  \endgroup
  \@lCProgfalse
}
\def\h@CProg#1;#2;{%
  \def\@CProg@a{#1:}\def\@CProg@b{#2}%
  \if@vCProg
    \fbox{\strut\expandafter\v@CProg@var\@CProg@a}%
  \else
    \fbox{\strut\expandafter\v@CProg\@CProg@a}%
  \fi
  \expandafter\expandafter\expandafter
  \ifx\expandafter\f@tok\@CProg@b\af@tok\relax
    \let\n@xt\relax
  \else
    \ \Arrowright\ %
    \let\n@xt\h@CProg\def\@CProg@b{#2;}%
  \fi
  \expandafter\n@xt\@CProg@b%
}%
\def\v@CProg#1:#2:{%
  \raise-.5\baselineskip\vbox{\hbox{#1}\hbox{#2}}%
}
\def\v@CProg@var#1:{#1}

%    \end{macrocode}
% \end{macro}
% \subsection{\MusicSymbols}
%    \begin{macrocode}
%%==============================================================================
%% 譜面上の記号
%%==============================================================================
%    \end{macrocode}
% \subsubsection{over/under square bracket}
% \paragraph{概要}
% \begin{macro}{\Jovbkt}
% \oarg{text}\marg{height}\marg{length}\oarg{ムニャ}\marg{slope}の形で
% 上付きの角括弧を作ります.
% \begin{macro}{\Junbkt}
% \oarg{text}\marg{height}\marg{length}\oarg{ムニャ}\marg{slope}の形で
% 下付きの角括弧を作ります.
% \begin{itemize}
% \item 第1引数(optional)は角括弧の真ん中,上付きなら上に・
%       下付きなら下に,引数に与えた文字を出します.
% \item 第2引数は角括弧の始点の高さ[\meta{exp\_note}]を与えます.
% \item 第3引数は角括弧の長さ[|noteskip|]を与えます.
% \item 第4引数(optional)はムニャ…….
% \item 第5引数は角括弧の始点からの相対的な終点の高さ[|internote|]を与えます.
% \end{itemize}
%
% \paragraph{実装}
% 内部的には後述する\cmd{\bell@bkt}を使っています.
% \cmd{\Jovbkt}と\cmd{\Junbkt}の違いは,内部で\cmd{\reserved@e}に与えている
% 数字の違いだけです.
%    \begin{macrocode}
%% over/under square bracket                                                  %%
%% \Jovbkt, \Junbkt #1=text(optional)   #2=height[\internote]                 %%
%%                  #3=length[noteskip] #4=slope [\internote]                 %%
\newcommand{\Jovbkt}[3][]{%
  \inhgetn@i#2\relax%
  \def\reserved@e{1}%
  \@ifnextchar[{\bell@bkt[#1]{\n@i}{#3}}%
               {\bell@bkt[#1]{\n@i}{#3}[0pt]}%
}
\newcommand{\Junbkt}[3][]{%
  \inhgetn@i#2\relax%
  \def\reserved@e{0}%
  \@ifnextchar[{\bell@bkt[#1]{\n@i}{#3}}%
               {\bell@bkt[#1]{\n@i}{#3}[0pt]}%
}

%    \end{macrocode}
% \end{macro}
% \end{macro}
% \begin{macro}{\bell@bkt}
% \oarg{text}\marg{height}\marg{length}\oarg{ムニャ}\marg{slope}の形で
%    \begin{macrocode}
%% bell@bkt #1=text(optional) #2=height[\internote] #3=length[noteskip]       %%
%%          #4=slope(differ)[\internote] #5=bkt over/under[+1/-1]             %%
\def\bell@bkt[#1]#2#3[#4]#5{%
  \y@ii#3\noteskip \y@iii#5\internote
  \advance\y@ii-.3\qn@width
  \advance\y@ii#4\relax
  \expandafter\ifnum\reserved@e>0\relax% bracket over
    \y@i\internote
    \settodepth{\y@iv}{#1}%
  \else% bracket under
    \y@i-\internote
    \settoheight{\y@iv}{#1}%
    \y@iv\m@ne\y@iv
  \fi
  \zchars{#2}{%
    \hspace{.65\qn@width}%
    \tikz[baseline=0.0ex]{\draw[semithick]
         (0,\y@i) -- ++(0,\y@i) -- ++(\y@ii,\y@iii) -- ++(0,-\y@i);}%
    \y@iii\h@lf\y@iii \advance\y@iii\tw@\y@i \advance\y@iii\y@iv
    \hspace{-.5\y@ii}\raisebox{\y@iii}{\lrlap{#1}}%
  }%
}

%    \end{macrocode}
% \end{macro}
% \subsubsection{left/right square bracket}
% 音符の左・右に角括弧を作ります.
% \begin{macro}{\Lbracket}
% \begin{macro}{\Rbracket}
% 上付き・下付きの角括弧を作ります.
% \begin{itemize}
% \item 第1引数は角括弧の下端の高さ[\meta{exp\_note}]を与えます.
% \item 第2引数は角括弧の上端の高さ[|internote|]を与えます.
% \end{itemize}
% 内部的には後述する|\bell@bracket|を使っています.
%    \begin{macrocode}
%% left/right square bracket                                                  %%
%% \Lbracket, \Rbracket                                                       %%
%% #1: bottom height[\internote], #2: height length[internote]                %%
\newcommand{\Lbracket}{\bell@bracket{\@ne}}
\newcommand{\Rbracket}{\bell@bracket{\m@ne}}
%    \end{macrocode}
% \end{macro}
% \end{macro}
% \begin{macro}{\bell@bracket}
%    \begin{macrocode}
\newcommand{\bell@bracket}[3]{%
  %\inhgetn@i#2\relax
  \y@iii#3\internote
  \ifnum#1>0\relax% bracket Left
    \lchars{#2}{%
      \tikz[baseline=0.0ex]{%
        \draw[semithick]
          (\qu@rt\qn@width,0) --++ (-\qu@rt\wn@width,0)
          --++ (0,\y@iii) --++(\qu@rt\wn@width,0);}%
      \hspace{\qu@rt\wn@width}%
    }%
  \else% bracket Right
    \zchars{#2}{%
      \hspace{1\qu@rt\wn@width}%
      \tikz[baseline=0.0ex]{%
        \draw[semithick]
          (0,0) --++ (\qu@rt\wn@width,0)
          --++ (0,\y@iii) --++(-\qu@rt\wn@width,0);}%
    }%
  \fi
}

%    \end{macrocode}
% \end{macro}
% \subsubsection{lines}
% \begin{macro}{\Mline}
% \oarg{add.~height}\marg{height}\oarg{add.~length}\marg{length}\marg{slope}の形で
% 線を引きます.
% 星付きのバージョンもあります.
% \begin{itemize}
% \item \meta{add.~height}は線の始点の高さのオフセット[\texttt{internote}]を与えます.
% \item \meta{height}は線の始点の高さ[\texttt{note pitch}]を与えます.
% \item \meta{add.~length}は追加の線の長さ[\texttt{dimen.}]を与えます.
% \item \meta{length}は線の長さ[\texttt{noteskip}]を与えます.
% \item \meta{slope}は線の終点の:
% \begin{itemize}
%   \item 線の始点からの相対的な高さ[\texttt{internote}]を与えます(星なし).
%   \item 絶対的な高さ[\texttt{internote}]を与えます(星あり).
% \end{itemize}
% \end{itemize}
% 内部的には後述する\cmd{\bell@MlineTikz}を使っています.
% \changes{v1.2}{2020/04/20}
%    {定義の\textbf{破壊的}変化.}
% 2020/04/20より前は\oarg{add\_length}\marg{height}%
%                   \marg{length}\marg{slope}の形でした
%    \begin{macrocode}
%% lines                                                                      %%
%% Mline  #1=height[\internote] #2=length[noteskip] #3=slope(diff.)[\internote]
%% Mline* #1=height[\internote] #2=length[noteskip] #3=slope [\internote]
\newcommand{\Mline}{\@ifstar{\bell@MlineTikzS}{\bell@MlineTikz}}
%    \end{macrocode}
% \end{macro}
% \begin{macro}{\MlineC}
% \oarg{add\_length}\marg{height}\marg{slope}\par
% \marg{length}に1を与えた変種として\cmd{\MlineC}を定義します.
%    \begin{macrocode}
\newcommand{\MlineC}{\@ifstar{\bell@MlineCTikzS}{\bell@MlineCTikz}}
%    \end{macrocode}
% \end{macro}
% \begin{macro}{\MlineB}
% \oarg{add\_length}\marg{height}\marg{slope}\par
% 同様にして,\marg{length}に1を与え,かつ\oarg{add\_length}に
% \cmd{\afterruleskip}を加える変種として\cmd{\MlineB}を定義します.
%    \begin{macrocode}
\newcommand{\MlineB}{\@ifstar{\bell@MlineBTikzS}{\bell@MlineBTikz}}

%    \end{macrocode}
% \end{macro}
% \begin{macro}{\bell@MlineTikz}
% \changes{v1.2}{2020/04/20}
%    {定義の\textbf{破壊的}変化.}
%    \begin{macrocode}
%% \bell@MlineTikz                                                            %%
%%   #1=additional height[\internote](opt.)                                   %%
%%   #2=height[pitch]                                                         %%
%%   #3=additional length[dimen](opt.)                                        %%
%%   #4=length[noteskip]                                                      %%
%%   #5=slope(differ) [\internote]                                            %%
\def\bell@MlineTikz{%
  \@ifnextchar[{\bell@MlineTikz@a}{\bell@MlineTikz@a[0]}%
}
%    \end{macrocode}
% \end{macro}
% \begin{macro}{\bell@MlineTikz@a}
%    \begin{macrocode}
\def\bell@MlineTikz@a[#1]#2{%
  \y@i#1\internote\relax
  \inhgetn@i#2\relax
  \advance\y@i\n@i\internote
  \y@iv\z@
  \@ifnextchar[{\bell@MlineTikz@b}{\bell@MlineTikz@b[\z@]}%
}
%    \end{macrocode}
% \end{macro}
% \begin{macro}{\bell@MlineTikz@b}
%    \begin{macrocode}
\def\bell@MlineTikz@b[#1]{%
  \y@ii#1\relax
  \bell@MlineTikz@c
}
%    \end{macrocode}
% \end{macro}
% \begin{macro}{\bell@MlineTikz@c}
%    \begin{macrocode}
\def\bell@MlineTikz@c#1#2{%
  \y@iii#1\noteskip
  \advance\y@iii\y@ii
  \advance\y@iv#2\internote
% line lenght offset of note head
  \advance\y@iii-1\thr@@qu@rt\qn@width
  \check@staff
    \raise\y@i\rlap{%
      \hspace{1\h@lf\qn@width}% to start drawing line at right of note head
      \tikz[baseline=0.0ex]{\draw[semithick] (0,0) -- (\y@iii,\y@iv);}%
    }%
  \fi
}
%    \end{macrocode}
% \end{macro}
% \begin{macro}{\bell@MlineTikzS}
%    \begin{macrocode}
\def\bell@MlineTikzS{%
  \@ifnextchar[{\bell@MlineTikzS@a}{\bell@MlineTikzS@a[0]}%
}
%    \end{macrocode}
% \end{macro}
% \begin{macro}{\bell@MlineTikzS@a}
%    \begin{macrocode}
\def\bell@MlineTikzS@a[#1]#2{%
  \y@i#1\internote\relax
  \inhgetn@i#2\relax
  \advance\y@i\n@i\internote
  \y@iv-\n@i\internote
  \@ifnextchar[{\bell@MlineTikz@b}{\bell@MlineTikz@b[\z@]}%
}
%    \end{macrocode}
% \end{macro}
% \begin{macro}{\bell@MlineCTikz}
%    \begin{macrocode}
\def\bell@MlineCTikz{%
  \@ifnextchar[{\bell@MlineCTikz@a}{\bell@MlineCTikz@a[0]}%
}
%    \end{macrocode}
% \end{macro}
% \begin{macro}{\bell@MlineCTikz@a}
%    \begin{macrocode}
\def\bell@MlineCTikz@a[#1]#2{%
  \y@i#1\internote\relax
  \inhgetn@i#2\relax
  \advance\y@i\n@i\internote
  \y@iv\z@
  \@ifnextchar[{\bell@MlineCTikz@b}{\bell@MlineCTikz@b[\z@]}%
}
%    \end{macrocode}
% \end{macro}
% \begin{macro}{\bell@MlineCTikz@b}
%    \begin{macrocode}
\def\bell@MlineCTikz@b[#1]{%
  \y@ii#1\relax
  \bell@MlineTikz@c{1}%
}
%    \end{macrocode}
% \end{macro}
% \begin{macro}{\bell@MlineCTikzS}
%    \begin{macrocode}
\def\bell@MlineCTikzS{%
  \@ifnextchar[{\bell@MlineCTikzS@a}{\bell@MlineCTikzS@a[0]}%
}
%    \end{macrocode}
% \end{macro}
% \begin{macro}{\bell@MlineCTikzS@a}
%    \begin{macrocode}
\def\bell@MlineCTikzS@a[#1]#2{%
  \y@i#1\internote\relax
  \inhgetn@i#2\relax
  \advance\y@i\n@i\internote
  \y@iv-\n@i\internote
  \@ifnextchar[{\bell@MlineCTikz@b}{\bell@MlineCTikz@b[\z@]}%
}
%    \end{macrocode}
% \end{macro}
% \begin{macro}{\bell@MlineBTikz}
%    \begin{macrocode}
\def\bell@MlineBTikz{%
  \@ifnextchar[{\bell@MlineBTikz@a}{\bell@MlineBTikz@a[0]}%
}
%    \end{macrocode}
% \end{macro}
% \begin{macro}{\bell@MlineBTikz@a}
%    \begin{macrocode}
\def\bell@MlineBTikz@a[#1]#2{%
  \y@i#1\internote\relax
  \inhgetn@i#2\relax
  \advance\y@i\n@i\internote
  \y@iv\z@
  \@ifnextchar[{\bell@MlineBTikz@b}{\bell@MlineBTikz@b[\z@]}%
}
%    \end{macrocode}
% \end{macro}
% \begin{macro}{\bell@MlineBTikz@b}
%    \begin{macrocode}
\def\bell@MlineBTikz@b[#1]{%
  \y@ii#1\relax
  \advance\y@ii\afterruleskip
  \bell@MlineTikz@c{1}%
}
%    \end{macrocode}
% \end{macro}
% \begin{macro}{\bell@MlineBTikzS}
%    \begin{macrocode}
\def\bell@MlineBTikzS{%
  \@ifnextchar[{\bell@MlineBTikzS@a}{\bell@MlineBTikzS@a[0]}%
}
%    \end{macrocode}
% \end{macro}
% \begin{macro}{\bell@MlineBTikzS@a}
%    \begin{macrocode}
\def\bell@MlineBTikzS@a[#1]#2{%
  \y@i#1\internote\relax
  \inhgetn@i#2\relax
  \advance\y@i\n@i\internote
  \y@iv-\n@i\internote
  \@ifnextchar[{\bell@MlineBTikz@b}{\bell@MlineBTikz@b[\z@]}%
}

%    \end{macrocode}
% \end{macro}
% \subsubsection{parentheses}
%    \begin{macrocode}
%% parentheses                                                                %%
%% Mparenl/Mparenr                                                            %%
%% #1=additional space[\wn@width](optional)                                   %%
%% #2=bottom height[\internote] #3=top height[\internote]                     %%
\newcommand{\Mparenl}[3][0]{%
  \y@i#2\internote
  \y@ii#3\internote
  \advance\y@ii-\y@i%
  \settodepth{\@tempdima}{(}%
  \settowidth{\y@iii}{(}%
  \y@i#1\wn@width \advance\y@i\h@lf\wn@width
  \lchar{#2}{\resizebox{\y@iii}{\y@ii}{\raisebox{\@tempdima}{(}}\hspace{\y@i}}%
}
\newcommand{\Mparenr}[3][0]{%
  \y@i#2\internote
  \y@ii#3\internote
  \advance\y@ii-\y@i%
  \settodepth{\@tempdima}{)}%
  \settowidth{\y@iii}{)}%
  \y@i#1\wn@width \advance\y@i1\h@lf\wn@width
  \zchar{#2}{\hspace{\y@i}\resizebox{\y@iii}{\y@ii}{\raisebox{\@tempdima}{)}}}%
}

%    \end{macrocode}
% \subsubsection{miscellaneous}
% \begin{macro}{\Mryaku}\oarg{height}\par
% 省略記号を出力します.
% \begin{itemize}
% \item 第1引数(opt.)は出力される高さ[|internote|]を与えます.
%       既定は3です.
% \end{itemize}
%    \begin{macrocode}
%% miscellaneous                                                              %%
\newcommand{\Mryaku}[1][3]{\shake{#1}}
%    \end{macrocode}
% \end{macro}
% \begin{macro}{\Loffset}\marg{offset}\marg{exp.}\par
% \begin{macro}{\Roffset}\marg{offset}\marg{exp.}\par
% 全音符単位の\marg{offset}[|number|]を指定します.
%    \begin{macrocode}
\def\Loffset#1{\edef\s@o@{#1\wn@width}\offs@t}
\def\Roffset#1{\edef\s@o@{-#1\wn@width}\offs@t}
%    \end{macrocode}
% \end{macro}
% \end{macro}
% \begin{macro}{\Roff}\marg{exp.}\par
% \begin{macro}{\Loff}\marg{exp.}\par
% \begin{macro}{\hRoff}\marg{exp.}\par
% \begin{macro}{\hLoff}\marg{exp.}\par
% それぞれ\cmd{\roff},\cmd{\loff},\cmd{\hroff}および\cmd{\hloff}の全音符版です.
%    \begin{macrocode}
\def\Roff{\Roffset\@ne}
\def\Loff{\Loffset\@ne}
\def\hRoff{\Roffset\h@lf}
\def\hLoff{\Loffset\h@lf}
%    \end{macrocode}
% \end{macro}
% \end{macro}
% \end{macro}
% \end{macro}
%    \begin{macrocode}
\def\qroff{\roffset\qu@rt}
\def\qloff{\loffset\qu@rt}
\def\qhroff{\roffset\thr@@qu@rt}
\def\qhloff{\loffset\thr@@qu@rt}
%    \end{macrocode}
% \begin{macro}{\doubleBAR}\par
% 終止線を出力します.
%    \begin{macrocode}
\def\doubleBAR{\setdoubleBAR\bar}
%    \end{macrocode}
% \end{macro}
% \begin{macro}{\Mhshift}\marg{offset}\marg{exp.}\par
% \meta{dimen}単位でオフセットする命令.
%    \begin{macrocode}
\newcommand{\Mhshift}[2]{\rlap{\hspace{#1\noteskip}#2\hspace{-#1\noteskip}}}

%    \end{macrocode}
% \end{macro}
% \subsection{\Resets}
%    \begin{macrocode}
%%==============================================================================
%% 初期設定
%%==============================================================================
%    \end{macrocode}
%    \begin{macrocode}
\smallmusicsize
\geometricskipscale
\setlength{\afterruleskip}{2\elemskip}%

\calc@accidentaloffs@tunit

\newlength{\flwidth}
\newlength{\shwidth}
\newlength{\nawidth}
\newlength{\dflwidth}
\newlength{\dshwidth}
\settowidth{\flwidth}{\musixchar50}
\settowidth{\shwidth}{\musixchar52}
\settowidth{\nawidth}{\musixchar54}
\settowidth{\dflwidth}{\musixchar51}
\settowidth{\dshwidth}{\musixchar53}
\let\QNwidth\qn@width
\let\WNwidth\wn@width

\def\raisevolta{\tw@\internote}
\def\gen@setvolta#1{\def\volta@text{\footnotesize #1}}
%</macro>
%    \end{macrocode}
%
% 以上で終わりです.
%    \begin{macrocode}
\endinput
%    \end{macrocode}
%
% \Finale
% \PrintIndex \PrintChanges
